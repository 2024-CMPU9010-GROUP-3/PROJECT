\subsection{Definition of Success}
Our project defines success by how effective it is in addressing the identified
challenges faced by Urban Planners.

\noindent{}To gauge if \textit{Magpie} solves the problems outlined in chapter 1
and 2, the following points need to be evaluated:
\vspace{-3mm}
\begin{itemize}
  \item{\textbf{Efficient Data Aggregation:} Does \textit{Magpie} offer the user
  a comprehensive overview of multiple kinds of public amenities in a
  geographical region they selected?}
  \item{\textbf{Automatic Data Extraction:} Does \textit{Magpie} include a data
  pipeline to extract and integrate new data automatically as it becomes
  available?}
  \item{\textbf{Accuracy of Public Amenities Identification:} Are at least 80\%
  of public amenities in the dataset correctly identified and are at the most
  10\% identified incorrectly?}
  \item{\textbf{Straightforward Interface:} Does the user interface allow for
  the retrieval of desired data efficiently (measured using Click Testing,
  including metrics such as: Time Till First Click, Total Time on Task, Overall
  Number of Clicks, and Ease of Finding/Clicking)?}
  \item{\textbf{Accessibility:} Does \textit{Magpie} make a reasonable effort to
  make its user interface accessible to users with special needs?}
\end{itemize}

% Adds page break to ensure blocking
\pagebreak{}

\subsection{Code Integrity}
To guarantee the integrity of \textit{Magpie's} code, we have devised a test
strategy using automated unit tests, integration tests and end-to-end tests.

\noindent{}Unit testing allows us to do granular regression testing, alerting us
to issues before they even enter the main codebase. Integration tests offer a
surefire way to confirm that each module is compiling correctly and is ready to
be released. End-to-end tests inspect the interaction of all components,
ensuring compatibility between modules.

\noindent{}For unit tests, we aim for 100\% code coverage where practical. In
the future, we want to adopt a test-driven development workflow to prevent any
code from going untested. In the case of breaking changes, we are discussing the
use of A/B-testing or canary testing.

\subsection{CI/CD}
Using a continuous integration/continuous delivery pipeline is industry standard
and allows for rapid iteration and shorter release cycles. Our automated testing
strategy is run by the pipeline before every public release, ensuring code
integrity. Automatic building and deployment allows us to offer users a stable
software solution that is always up-to-date and regularly includes new features.