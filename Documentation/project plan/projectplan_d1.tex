\documentclass[a4paper,12pt]{article}
\usepackage{graphicx} % For including logos or images
\usepackage{array}    % For table management
\usepackage{geometry} % To set page margins
\usepackage{enumitem} % To customize bullet points
\usepackage{booktabs}
\usepackage{tabularx}


% Set the margins
\geometry{top=2cm, bottom=2cm, left=2.5cm, right=2.5cm}

\begin{document}

% Add the logo and title at the top
\begin{center}
    \includegraphics[width=3cm]{tud logo header.png}\\
    \vspace{0.3cm}
    {\Large \textbf{MSc in Computer Science - Team Project}}\\
    \vspace{0.2cm}
    {\large \textbf{Project Plan}}
    \vspace{1cm}
\end{center}

% Center the first table and adjust column widths
\centering
\begin{tabular}{|p{0.25\textwidth}|p{0.65\textwidth}|}
    \hline
    \textbf{Project Title:} & \parbox{0.65\textwidth}{\vspace{0.3cm}Utilizing Computer Vision for Service Location Determination\vspace{0.3cm}} \\
    \hline
    \textbf{Project Summary:} & \parbox{0.65\textwidth}{\vspace{0.3cm}\textbf{ What are you doing? }\vspace{0.3cm}
    \\The aim of this project is to create a Single Page Application (SPA) to aid urban design decisions by providing detailed insight into available amenities which include and are not limited to parking spaces both street-level and underground, bike sheds, accessible ramps, bike lanes, etc.\\
    The SPA will integrate computer vision with publicly available mapping datasets.
    \\
    \\
    \vspace{0.3cm}\textbf{Why are you doing it?} \vspace{0.3cm}
    \\To help enhance the efficiency of urban planning for public authorities by providing detailed insight into available amenities.
    \\
    \\
    \vspace{0.3cm}\textbf{Who will use it? And how will they use it? (Example use case)} \vspace{0.3cm}
    \\The main users of our application are civil servants and private individuals.
    \\
    \\Civil servants will use the app  for urban planning purposes such as in the following use cases:
    \begin{itemize}[label=\textbullet]
            \item to ensure parking spaces adhere to local zoning regulations
            \item to assess the demand/supply for parking in various areas of a city
            \item to oversee existing parking amenities, conditions and pricing
            \item to consider environmental and traffic impact when planning to build new spaces
        \end{itemize}
    \\Private individuals will use the app in the following use cases:
    \begin{itemize}[label=\textbullet]
            \item to view real-time parking availability given their current location or a specified destination within a defined radius
            \item to see the cost of available parking spots nearby allowing them to compare and choose the best option
            \item to view available parking based on special filters, such as handicap-accessible parking, electric vehicle charging stations, etc...
    \end{itemize}} \\
    \hline
\end{tabular}    
\begin{tabular}{|p{0.25\textwidth}|p{0.65\textwidth}|}
    \hline
    \textbf{Project Development:} & \parbox{0.65\textwidth}
    {\vspace{0.3cm}The project development will be done following the SCRUM methodology.\vspace{0.3cm}
    \\
    The first minimal viable product (MVP) will include the following components :
    \begin{itemize}[label=\textbullet]
        \item 
        \item
        \item 
    \end{itemize}
    
    \vspace{0.3cm}\textbf{What will you learn from deploying the first MVP and what would you measure in order to learn it.}\vspace{0.3cm}
    \\
    text
    \\
    \\
    \vspace{0.3cm}\textbf{How will you build your system ? (System diagram)}\vspace{0.3cm}
    \\
    diagram 1
    \\
    \\
    \vspace{0.3cm}\textbf{Front-end: User interface components}\vspace{0.3cm}
    \\
    diagram
    \\
    \\
    \vspace{0.3cm}\textbf{ Back-end: Technical components}\vspace{0.3cm}
    \\
    diagram
    \\
    \\
    \vspace{0.3cm}\textbf{Data sources : What data will you use and how will you access it?}\vspace{0.3cm}
    \\
    We will use satellite data obtained using the Google Static Maps API. Training, testing and validation datasets will be collected and manually annotated to train and evaluate our ML model. For each user request, data will be collected on the fly based on the location and radius provided by the user.
    \\}
    \\
   \hline
\end{tabular}

\begin{tabular}{|p{0.25\textwidth}|p{0.65\textwidth}|}
    \hline
   \textbf{Evaluation:} & \parbox{0.65\textwidth}{\vspace{0.3cm}\textbf{How will you evaluate your system? (Initial ideas)} \vspace{0.3cm}
   To evaluate our system we will get in touch with a group of users who will be given access to our application for testing and feedback purposes. Moreover, our application will undergo functionality and performance testing to ensure it meets the required standard.
   \\
   To evaluate the Machine Learning model, we will use a separate validation dataset ensuring it performs well in different urban settings. Furthermore several performance metrics (accuracy, precision, recall) will be used to assess the model.
   \\
   \\
   \vspace{0.3cm}\textbf{ How will the evaluation inform your future development?}\vspace{0.3cm}
   \\
   User feedback will help the development of additional features suggested by the users and identifying potential bugs and issues.
   \\
   \\
    } \\
   \hline
\end{tabular}

\vspace{0.5cm}

\centering
\begin{tabular}{|p{0.25\textwidth}|p{0.65\textwidth}|}
    \hline
    \textbf{Project Management:} & \parbox{0.65\textwidth}{\vspace{0.3cm}
    We are using Github and their built-in project management tool. We are also adopting an Agile approach where we'll be working on completing milestones every week.\\ \\
    Deadlines will include project initialization, first mock-up, dataset creation, model development and training, testing phase and final submission. \\ \\
    Success in this project is good teamwork all throughout and delivering a good quality working product we are all satisfied with.
    \vspace{0.3cm}} \\
    \hline
\end{tabular}

\newpage
\section{Team information}
\textbf{Team Name:} \textbf{GROUP 3}
\vspace{0.2cm}
\begin{table}[h]
    \centering
    \begin{tabularx}{\textwidth}{@{}p{4cm} p{4cm} p{5cm}@{}}
        \toprule Student Name & Student Number & Email \\
        \midrule
            Jessica Fornetti & D23124588 & D23124588@mytudublin.ie \\ \\
            Anais Blenet & D22127697 & D22127697@mytudublin.ie \\ \\
            Saul Burgess & C19349793 & C19349793@mytudublin.ie \\ \\
            Kaustubh Trivedi & D23124940 & D23124940@mytudublin.ie \\ \\
            Yuanshuo Du & D22125495 & D22125495@mytudublin.ie \\ \\
            Andreas Kraus & D23125112 & D23125112@mytudublin.ie \\ 
        \bottomrule 
    \end{tabularx}
\end{table}

\begin{tabular}{|p{0.25\textwidth}|p{0.65\textwidth}|}
    \hline
    \textbf{Team Meetings:} & \parbox{0.65\textwidth}
    {\vspace{0.3cm}
    We have decided to meet everyday online at 12pm, to check in for the day, discuss previous day and current day tasks. \\ These online meetings last around 30 minutes and everyone is usually in attendance except of Kaustubh who works night shifts. Saul keeps minutes of each meetings and updates Kaustubh subsequently. \\ \\ We have also decided to meet in person on Fridays after the presentation. \\ \\ Decision-making so far is unanimous, we'll see how that progresses. Turn-taking we still need to discuss as each member has their own expertise. \vspace{0.3cm}} \\
    \hline
\end{tabular}

% Center the third table and adjust column widths
\centering
\vspace{0.2cm}
\begin{tabular}{|p{0.25\textwidth}|p{0.65\textwidth}|}
    \hline
   \textbf{Team Conflict:} & \parbox{0.65\textwidth}{\vspace{0.3cm}The big keyword here is communication. We are set up on a discord server with multiple breakout rooms and we are constantly updating each other on what we are doing/not doing/able to do/unable. This will help avoid conflict often caused by miscommunication. \\ \\ In the event of conflict, one of us will act as mediator to oversee the issue and talk it out. Issues shall not be left unresolved. \\ \\ Ultimate veto power seems excessive. \vspace{0.3cm}} \\
   \hline
\end{tabular}

\end{document}
