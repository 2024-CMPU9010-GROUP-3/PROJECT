In the future, one of the major focuses on the frontend development of Magpie will be the implementation of user feedback and the improvement of the look and feel of the platform. The current interface provides a great skeleton, but user interaction throughout the platform and accuracy of data displayed need a little more work.

\subsubsection{Enhance Visual feedback}

One of the main areas of improvement is in providing better \textbf{visual feedback} for certain features. Specifically, users have asked for more dynamic responses while interacting with the \textbf{search feature} and the \textbf{amenity selection/deselection}. Implementation of animations or progress indicators will make the process smoother and more intuitive, especially when users are filtering or navigating through large data. This could be followed by increasing the \textbf{precision} of the \textbf{points shown on} the map. The increased accuracy will give users a better place{-}marking experience and may also go on to increase the reliability of the information provided.

\subsubsection{Integrate User Feedback}

The feedback from these \textbf{target users} has been gold. In the user interviews, each urban planner brings varied experiences and a need for different kinds of requirements; hence, every single approach became relevant.

\subsubsection{More Data}

This was a clear indication of the need for feature customization to suit these diverging expectations. Hence, further work will involve the inclusion of \textbf{more data} in the tool in order to give a richer and more detailed view of amenities. Moreover, the \textbf{tooltip information} of each amenity will be enhanced with more specific and actionable data that will make the tool useful for more detailed urban planning tasks.

\subsubsection{Export Data}

Data export functionality There has also been interest expressed in an \textbf{export feature} to take the data off{-}line and further analyze them. That will be also prioritized, so that planners who do need to make presentations of their findings or pass on some of their research can really use this tool much more effectively.