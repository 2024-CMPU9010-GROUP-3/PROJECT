\subsubsection{Application Security}
\textbf{Fix Issues with JWTs}\\
As mentioned in the discussion of JSON Web Tokens on page \pageref{jwt}, the use
of JWTs comes with its own set of caveats.

The most pressing of these issues is the missing token revocation functionality.
One solution for these issues would be implementing a list of revoked tokens on
the server and offering the user a way to add a stolen token to that list. If an
attacker then uses one of these revoked tokens, their requests can be easily
denied.

\textbf{Implement Email-based Features}\\
Right at the end of the project, the team experimented with email-based features
like email verification and password reset. Unfortunately, emails sent by our
system were always classified either classified as spam or didn't get delivered
at all as they were considered untrustworthy. Since the deadline was fast
approaching, the team decided to abandon this idea and focus on other issues.

Nonetheless, these features are essential components for any public facing web
application. This is why one of the next steps would be to implement email
verification and password reset using a service like SendGrid. This way, the
issue of emails being blocked automatically should be mitigated.

\textbf{Replacing bcrypt}\\
Switching out bcrypt for Argon2id is not as pressing as other issues on this
list as it is still functional and secure, but sometime in the future it would
be a good idea to switch to a newer password encryption solution. This would
bring Magpie in line with the current recommendations for secure password
storage by the \textcite{owasp_password_storage_cheatsheet}.

Replacing the bcrypt library would make it necessary to reset the passwords of
all users, as it is not possible to convert a bcrypt hash to a Argon2id hash.
This should therefore be done before the number of users grows to affect as few
people as possible.

The issue of improving application security was kept in mind during the whole
development process, but other issues always took precedence. Given more time,
improving the application security would be one of the top priorities in terms
of further backend development.

\subsubsection{Performance}
\textbf{Improve Data Transmission Speed}\\
Currently, when a user selects a large area of the map it can take a couple of
seconds from when they place the circle until the points appear on the map. This
is due the data being sent in one big chunk which then needs to be processed
completely before the points can be shown. This is not ideal and better
responsiveness could improve the user experience.

To achieve this, the data would need to be transmitted in smaller chunks. This
would allow the frontend to start processing and showing the points even while
not all of them are loaded. While it would still take at least the same time to
transmit all of the data, the user would be presented with some data earlier,
making the application feel much more responsive.

One technical solution to this problem would be streaming the data between
backend and frontend. This could be implemented using Server{-}sent Events,
which is a one{-}way data transfer solution that allow progressive delivery of
large amounts of data. Alternatively, a WebSocket connection could be used to
achieve a similar result.

\textbf{Database Optimisation}\\
As discussed in the section on database development on page
\pageref{database_development}, the database is not considered to be Magpie's
biggest bottleneck. As such and due to limited time, other issues took
precedence, so there is still room for optimisations in the database.

A straightforward optimisation would be to add a geospatial index to the points
table. Using geospatial indexing would help the database eliminate many rows
early in the querying process, leading to a much more efficient search. Some
sources achieved a six times speed increase without any other optimisations
\cite{postgis_indexing}. The team recognises that such an index should have been
added early in development, as it would reduce retrieval time and could make the
application as a whole more responsive.

The query used to retrieve the points in a radius could also benefit from some
optimisation. Currently, it uses the PostGIS \texttt{ST\_DWithin} function to
find points in a radius. To reduce the number of rows that the query needs to
compare, a bounding box filter could be used before calling
\texttt{ST\_DWithin}. The query currently casts the \texttt{LongLat} column of
type \texttt{geometry} to the \texttt{geography} type for every row on each
request. Converting the point to \texttt{geography} is necessary for calculating
the distance in meters, but it's possible to pre{-}calculate and store both
types in the table to eliminate unnecessary computation.

\subsubsection{Fully Automated Data Ingress}
At the moment, new data is added to the system by manually running a Python
script. This works for the current scope of the project, but should Magpie's
service area be expanded in the future this would not be sustainable. Therefore,
a fully automated solution should be put in place. This would take the form of a
container that has facilities to add new data to the database, either when new
datasets become available or upon user request.

For example, if a user queries an area for parking spots that is not currently
covered by Magpie, the backend could send a request to the data container which
would then fetch, classify and insert the data. While this approach would be too
slow to provide the data in a timely manner to the user who requested it, it
could still be stored in the database for when the next user requests it. There
could even be a function where users get a notification via email when an area
they queried has data added to it.

One challenge with this feature would be figuring out where new datasets overlap
with old datasets. Since the points for parking are generated by a machine
learning model, it is possible that point coordinates for the same parking spot
might differ slightly between runs. It's possible to just discard points with
the exact same coordinates, but there is no guarantee that no parking spot will
be counted multiple times.

A different problem arises with external datasets (such as from
\texttt{gov.ie}). While the coordinates of amenities such as public bins will
not change much, it is not safe to assume that the format of the dataset will
not change between releases. Designing an automated data ingress solution to
handle arbitrary data formats correctly would probably be a project in itself.

Solving these problems would enable Magpie to serve a larger area with less
reliance on manual input.