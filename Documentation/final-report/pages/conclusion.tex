The \textbf{Magpie} project successfully addresses the challenges faced by urban
planners and general users in accessing and analysing public amenities. By
leveraging modern technologies, such as machine learning and advanced
visualization tools, the system automates the identification and aggregation of
real-world data. This eliminates the need for manual data entry and provides an
intuitive, easy-to-use platform for informed decision-making.

The technical solutions, including the use of YOLO models for object detection
and robust system architectures, ensure high accuracy, scalability, and
efficiency. Additionally, the user-friendly frontend and the comprehensive
backend enable seamless interactions with the data while maintaining
accessibility for users with varying levels of technical expertise.

This report has demonstrated the innovative approaches employed in developing
\textbf{Magpie}, overcoming challenges like data fragmentation, low-resolution
imagery, and data misalignment. User evaluations and expert reviews validate the
system’s effectiveness and usability in real-world scenarios.

In the future, \textbf{Magpie} can be extended with enhanced machine learning
capabilities, additional data sources, and improved deployment strategies to
further scale its impact. The project lays a strong foundation for the continued
development of tools that promote sustainable, data-driven urban planning and
infrastructure management.
