The increasing demands of urbanization necessitate innovative solutions to
streamline city planning and promote sustainability. \textbf{Magpie} represents
a novel geographical information system that automates the identification,
aggregation, and visualization of public amenities. By addressing the
fragmentation of traditional data sources and reducing reliance on manual
methods, Magpie provides urban planners with actionable insights, enhancing
decision-making processes and infrastructure analysis.

\subsection{Project Management Strategy}
The project was developed following an agile methodology, which enabled
iterative design, development, and evaluation. Weekly sprint cycles and
continuous integration practices ensured consistent progress and adaptive
responses to challenges. Task management through GitHub, combined with pull
requests and structured code reviews, promoted team collaboration and
accountability. This methodology facilitated the alignment of individual
contributions with the overarching project goals.

\subsection{Challenges and Solutions}
The team encountered several key challenges during the project lifecycle:
\begin{itemize}
    \item \textbf{Data Quality and Extraction:} Low-resolution satellite
    imagery, seasonal variations, and occlusion presented obstacles to accurate
    object detection. To address this, the YOLOv8-OBB model was employed,
    offering high accuracy in detecting and classifying public amenities.
    \item \textbf{Scalability and Performance:} Processing large-scale datasets
    required optimization of machine learning pipelines and backend systems. By
    fine-tuning thresholds and leveraging efficient clustering algorithms such
    as DBSCAN, performance bottlenecks were mitigated.
    \item \textbf{Team Coordination:} The interdisciplinary nature of the
    project demanded effective communication and knowledge sharing across team
    members. Regular meetings and clearly defined roles fostered synergy and
    alignment toward common objectives.
\end{itemize}

\subsection{Key Contributions}
The primary contribution of this project lies in the development of a robust
system capable of automating the detection and analysis of public
infrastructure. Unlike existing GIS tools that often require significant manual
effort or advanced expertise, Magpie combines machine learning, data fusion, and
interactive visualizations to deliver a user-friendly platform.

Future work will focus on extending the capabilities of Magpie:
\begin{itemize}
    \item \textbf{Enhanced Data Integration:} Incorporating real-time data
    streams from IoT sensors, traffic systems, and environmental indicators to
    improve the accuracy and relevance of insights.
    \item \textbf{Object Detection Expansion:} Refining models to include
    additional infrastructure, such as bus stops, pedestrian zones, and green
    spaces, thereby broadening the scope of analysis.
    \item \textbf{Customization Features:} Introducing user-defined parameters
    for region-specific queries and enabling multi-layered data overlays for
    advanced analysis.
\end{itemize}

\subsection{Lessons Learned}
The development of Magpie provided significant insights into the complexities of
urban infrastructure analysis and machine learning applications:
\begin{itemize}
    \item The quality and reliability of data, particularly from satellite
    imagery, pose inherent challenges that require advanced preprocessing and
    model optimization techniques.
    \item Automation of data extraction and visualization can dramatically
    streamline the planning process, highlighting the potential of AI-driven
    tools in real-world scenarios.
    \item Effective project management, collaboration, and adaptability are
    essential when addressing multifaceted technical and organizational
    challenges.
\end{itemize}

\subsection{Final Remarks}
This report has outlined the key technical, managerial, and analytical
components of Magpie, emphasizing its contribution to urban planning through
automated detection and data visualization. While significant progress has been
achieved, the proposed extensions highlight avenues for further improvement. By
bridging the gap between fragmented data systems and practical decision-making
tools, Magpie represents a step toward accessible, data-driven, and sustainable
urban development.

\newpage{}