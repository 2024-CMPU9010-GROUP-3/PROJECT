The increasing demands of urbanization necessitate innovative solutions to
streamline city planning and promote sustainability. \textbf{Magpie} represents
a novel geographical information system that automates the identification,
aggregation, and visualization of public amenities. By addressing the
fragmentation of traditional data sources and reducing reliance on manual
methods, Magpie provides urban planners with actionable insights, enhancing
decision-making processes and infrastructure analysis.

\subsection{Project Management Strategy}
The project was developed following an agile methodology, which enabled
iterative design, development, and evaluation. Weekly sprint cycles and
continuous integration practices ensured consistent progress and adaptive
responses to challenges. Task management through GitHub, combined with pull
requests and structured code reviews, promoted team collaboration and
accountability. This methodology facilitated the alignment of individual
contributions with the overarching project goals.

\subsection{Challenges and Solutions}
The team encountered several key challenges during the project lifecycle:
\begin{itemize}
    \item \textbf{Labelling Software:} Finding an image labelling software was
          an initial hurdle. Ultimately, Label Studio, a popular labelling software
          was chosen for its easy to use interface and easily exported labels.
          \vspace{0.2cm}
          
    \item \textbf{Data Quality and Extraction:} Low-resolution satellite
          imagery, seasonal variations, and occlusion presented obstacles to accurate
          object detection. To address this, the YOLOv8-OBB model was employed,
          offering high accuracy in detecting and classifying public amenities.
          \vspace{0.2cm}
          
    \item \textbf{Scalability and Performance:} Processing large-scale datasets
          required optimization of machine learning pipelines and backend systems. By
          fine-tuning thresholds and leveraging efficient clustering algorithms such
          as DBSCAN, performance bottlenecks were mitigated.
          \vspace{0.2cm}
          
    \item \textbf{Evaluation of the ML submodules:} The careful selection and
          accurate labelling of each test set was key in correctly evaluating our
          model's performance.
          \vspace{0.2cm}
          
    \item \textbf{Team Coordination:} The interdisciplinary nature of the
          project demanded effective communication and knowledge sharing across team
          members. Regular meetings and clearly defined roles fostered synergy and
          alignment toward common objectives.
\end{itemize}

\newpage{}

\subsection{Key Contributions}
The primary contribution of this project lies in the development of a robust
system capable of automating the detection and analysis of public
infrastructure. Unlike existing GIS tools that often require significant manual
effort or advanced expertise, Magpie combines machine learning, data fusion, and
interactive visualizations to deliver a user-friendly platform.

Future work will focus on extending the capabilities of Magpie:
\begin{itemize}
    \item \textbf{Enhanced Data Integration:} Incorporating real-time data
          streams from IoT sensors, traffic systems, and environmental indicators to
          improve the accuracy and relevance of insights.
          \vspace{0.2cm}

    \item \textbf{Object Detection Expansion:} Refining models to include
          additional infrastructure, such as bus stops, pedestrian zones, and green
          spaces, thereby broadening the scope of analysis.
          \vspace{0.2cm}

    \item \textbf{Customization Features:} Introducing user-defined parameters
          for region-specific queries and enabling multi-layered data overlays for
          advanced analysis.
\end{itemize}

\subsection{Lessons Learned}
The development of Magpie provided significant insights into the complexities of
urban infrastructure analysis and machine learning applications:
\begin{itemize}
    \item The quality and reliability of data, particularly from satellite
          imagery, pose inherent challenges that require advanced image processing and
          model optimization techniques.
          \vspace{0.2cm}

    \item Automation of data extraction and visualization can dramatically
          streamline the planning process, highlighting the potential of AI-driven
          tools in real-world scenarios.
          \vspace{0.2cm}

    \item Effective project management, collaboration, and adaptability are
          essential when addressing multifaceted technical and organizational
          challenges.
          \vspace{0.2cm}

    \item Teamwork encourages learning from each other and sharing knowledge, allowing
          members to rely on each other's strengths and overcome challenges as a team.
          \vspace{0.2cm}

    \item Clear communication as a team and iterative development were key to implementing
          user feedback and meeting our project requirements.
\end{itemize}

\subsection{Strengths and Weaknesses of Final System}
Magpie's strengths and weaknesses provide both an overview of the milestones
achieved and the limitations encountered during project development.

Our system's strengths relate to the core completed objectives set out at
project inception:
\begin{itemize}
    \item \textbf{Ease of use} - the system features an intuitive interface that
          has gone through several iterations thanks to the valuable user feedback.
          Both domain experts, non-experts and UI/X \& accessibility professionals
          cited the simplicity of Magpie's interface, placement of control elements
          and visual appeal.
          \vspace{0.2cm}

    \item \textbf{Accessibility} - the user interface design prioritised
          accessibility to ensure users with varying levels of technological
          proficiency, English language and accessibility concerns. Damian Gordon
          evaluated our system and gave it an incredibly high score, allowing us to
          complete this objective.
          \vspace{0.2cm}

    \item \textbf{High-level visualisation} - the system visualises more than a
          dozen, complex, fragmented data types into a clear, sleek map visualisation
          that has been complimented all throughout the user evaluation cycle.
\end{itemize}

The limitations encountered were a result of the novel techniques undertaken for
the car parking detection as well as technical components misbehaving:
\begin{itemize}
    \item \textbf{Time \& Computational intensity} - all the steps of the
          machine learning section of the project were both extremely time \&
          computational intensive. The YOLO model used for the car detection required
          significant amount of training time, limiting the speed with which the
          development could progress. The road mask required several adjustments and
          testing sessions to improve model metrics, and the classification of spots
          revealed core issues in the model's limitations.
          \vspace{0.2cm}

    \item \textbf{System response time} - the time it takes for amenity data to
          load on the map is not as fast as we would've like. This is due to the
          method implemented for data retrieval which is not suited for the large
          amount of data being requested.
\end{itemize}

The system excels in several key areas including the ease of use and
accessibility of the user interface which were key to the project's goals.
However, some limitations arose due to the novel machine learning techniques and
the method implemented for data retrieval to the frontend.

\subsection{Final Remarks}
This report has outlined the key technical, managerial, and analytical
components of Magpie, emphasizing its contribution to urban planning through
automated detection and data visualization. While significant progress has been
achieved, the proposed extensions highlight avenues for further improvement. By
bridging the gap between fragmented data systems and practical decision-making
tools, Magpie represents a step toward accessible, data-driven, and sustainable
urban development.

\newpage{}