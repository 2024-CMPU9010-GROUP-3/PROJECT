\subsubsection{Frontend Architecture and Implementation}
The frontend of the Magpie project is central to delivering a seamless, responsive, and user-friendly interface for urban planners and casual users. With a focus on providing aggregated data on public amenities, the frontend not only caters to analytical needs but also ensures inclusivity, accessibility, and scalability. This section delves into the technologies, architectural decisions, and user-centric features that make Magpie’s frontend robust and future-ready.

\begin{enumerate}
    \item \textbf{Core Technologies and Frameworks:}
    \begin{enumerate}
        \item \textbf{React and Next.js:} \\
        \begin{itemize}
            \item \textbf{React} serves as the backbone of the user interface, enabling a component-driven architecture. This approach ensures that UI components are modular, reusable, and easy to maintain.
            \item \textbf{Next.js}, a React framework, introduces server-side rendering (SSR) and static site generation (SSG). These features improve page load times, enhance SEO performance, and deliver a seamless user experience. For instance, the SSR capability ensures that essential data for urban planners, such as maps and analytics, loads efficiently on the client side.
            \item The use of Next.js's routing capabilities facilitates dynamic navigation within the application, ensuring smooth transitions between different sections, such as data visualization dashboards and user account management pages.
        \end{itemize}
        \item \textbf{TailwindCSS}
        A utility-first CSS framework, TailwindCSS, provides a consistent design language across the application. Its predefined utility classes simplify styling while allowing for quick prototyping. 
        \textbf{Enhancements with TailwindCSS:: }
        \begin{itemize}
            \item \textbf{Rapid Development:}  Predefined classes reduce the need for custom CSS.
            \item \textbf{Customization:}  The framework supports themes and dark/light modes, catering to diverse user preferences.
            \item \textbf{Animation:}  TailwindCSS-animate introduces subtle, engaging animations, enhancing user interactions.
        \end{itemize}
        \item \textbf{Mapbox GL and React Map GL:} \\
        Mapping and geospatial data are the heart of Magpie. The integration of Mapbox GL and React Map GL enables interactive, feature-rich maps. Deck.gl adds advanced visualization capabilities, such as heatmaps and clustered data points.\\
        \textbf{Mapping Features:}
        \begin{itemize}
                \item \textbf{Dynamic Data Layers:} Users can toggle layers to focus on specific amenities like parking or bike paths.
                \item \textbf{Geospatial Analysis:} Heatmaps and visual overlays offer insights into the density and distribution of public amenities.
        \end{itemize}
        \item \textbf{ShadCN/UI and Radix Primitives:} \\
        For a consistent and accessible component library, Magpie employs ShadCN/UI, built on Radix Primitives. This integration ensures compliance with accessibility standards while enabling highly customizable UI elements.\\
        \textbf{Accessibility Focus:}
        \begin{itemize}
            \item \textbf{ARIA Labels:} Enhance screen reader compatibility.
            \item \textbf{Keyboard Navigation:} Interactive elements are accessible via keyboard.
            \item \textbf{Color Contrast:} Adherence to WCAG standards ensures readability.
        \end{itemize}
    \end{enumerate}
    \item \textbf{Frontend Architecture and Design Patterns:}\\
    Magpie’s frontend is designed to be scalable, maintainable, and user-focused. The architecture adheres to modern design principles, including separation of concerns, state management, and performance optimization.\\
    \begin{enumerate}
        \item \textbf{Component Driven Architecture:}\\
        Every UI element is treated as a standalone, reusable component. This modularity reduces development effort and ensures consistency across the application.\\
        \textbf{Example of Key Components:}
        \begin{itemize}
            \item \textbf{Map Interface:} Integrates Mapbox GL for geospatial visualization.
            \item \textbf{Filters and Search:} Dynamically updates the map based on user input.
            \item \textbf{Data Widgets:} Present aggregated statistics on amenities.
        \end{itemize}
        \item \textbf{State Management:}\\
        React’s Context API and hooks are used for efficient state management, reducing the need for external libraries like Redux. This keeps the codebase lightweight and focused.
        \textbf{State Management Use Cases:}
        \begin{itemize}
            \item \textbf{User Preferences:} Tracks theme settings, saved filters, and user-specific configurations.
            \item \textbf{API Integration:} Manages asynchronous data fetching from the backend.
            \item \textbf{Real-Time Updates:} Enables dynamic updates when new data is loaded.
        \end{itemize}
        \item  \textbf{API Integration:}
        The frontend communicates with a \emph{Go} backend via \emph{REST APIs} to fetch, update, and display data. Authentication is handled through JSON Web Tokens (JWT), ensuring secure access.
    \end{enumerate}
    \item \textbf{Data Management and State Handling:}\\
    Efficient state management is critical for a responsive frontend application. Magpie employs a combination of React Query and Zod to handle data fetching, validation, and state management.\\
    \begin{enumerate}
        \item \textbf{React Query:}\\
        \begin{itemize}
            \item React Query simplifies the process of fetching, caching, and synchronizing data between the client and backend. It ensures that users always have access to the latest data without unnecessary reloads.
            \item Features such as background updates and optimistic rendering enhance the user experience by providing real-time feedback during interactions, such as updating map layers or filtering amenities.
        \end{itemize}
        \item \textbf{Data Validation with Zod:}\\
        \begin{itemize}
            \item \textbf{Zod} a TypeScript-first schema declaration and validation library, ensures the integrity of data passed between the frontend and backend. By validating API responses and user inputs, Zod prevents errors and enhances the platform's robustness.
        \end{itemize}
        \item  \textbf{API Integration:}
        The frontend communicates with a \emph{Go} backend via \emph{REST APIs} to fetch, update, and display data. Authentication is handled through JSON Web Tokens (JWT), ensuring secure access.
    \end{enumerate}
    \item \textbf{User-Centric Features and Functionalities:}
    \begin{enumerate}
        \item \textbf{Interactive Maps:}\\
        The interactive map interface is at the heart of Magpie, offering dynamic, user-friendly visualization tools to meet the needs of both expert users like urban planners and casual users seeking localized information. Users can:
        \begin{itemize}
            \item \textbf{View Amenities:} Quickly toggle between layers for parking, bike infrastructure, public transport stops, and other amenities, providing a clear, categorized overview of available resources.
            \item \textbf{Analyze Data:} Heatmaps and clustering options enable users to identify areas of low or high amenity density, offering insights for urban planning or resource allocation.
            \item \textbf{Custom Views:} Save personalized configurations, such as predefined radii or selected amenities, for recurring analysis or quick future access.
            \item \textbf{Interactive Elements:} Users can click on markers for detailed information about specific amenities, including types, availability, and proximity to other key locations.
        \end{itemize}
        \item \textbf{Accessibility and Inclusivity:}\\
        Accessibility is a core design principle of Magpie, ensuring the platform is inclusive for all users:
        \begin{itemize}
            \item \textbf{Screen Reader Support:} ARIA labels and meaningful descriptions provide complete compatibility with assistive technologies, enabling visually impaired users to navigate and interact with the platform effectively.
            \item \textbf{Keyboard-Friendly Navigation:} All interactive elements are fully operable using keyboard shortcuts, ensuring users with motor impairments can access the platform seamlessly.
            \item \textbf{Responsive Design:} The interface dynamically adjusts to various screen sizes and resolutions, making it equally accessible on smartphones, tablets, and desktops.
            \item \textbf{Tutorial Guidance:} An onboarding experience guides first-time users, offering step-by-step instructions to ensure inclusivity even for non-technical audiences.
        \end{itemize}
        \item \textbf{Performance Optimization:}\\
        Magpie delivers a high-performance experience tailored for quick data access and seamless user interaction:
        \begin{itemize}
            \item \textbf{Lazy Loading:} Data and map layers are loaded incrementally, prioritizing visible content and reducing initial load times, even for data-intensive regions.
            \item \textbf{SSR and SSG:} Server-side rendering (SSR) and static site generation (SSG) optimize the platform’s response times and enable fast, efficient navigation, particularly for large datasets or slow network connections.
            \item \textbf{Caching Strategies:} Frequently accessed data is cached locally, minimizing redundant server requests and ensuring faster access during repeated visits. This also helps reduce latency for users operating in areas with limited connectivity.
            \item \textbf{Error Handling:} Advanced error prevention mechanisms guide users through potential pitfalls, such as incorrect radius adjustments or unavailable data points.
        \end{itemize}
        These features combine to provide a smooth, intuitive, and robust experience for diverse user groups, ensuring Magpie remains a valuable tool for urban analysis and decision-making.
    \end{enumerate}

\end{enumerate}

\subsubsection{Future Roadmap for Magpie Based on User Feedback}
Based on the detailed usability and field-testing feedback provided by various users, the following additions to the future roadmap for the Magpie frontend are proposed. These additions focus on addressing usability challenges, enhancing functionality, and broadening the system’s appeal to professional and casual users.

\begin{enumerate}
    \item \textbf{Advanced Mapping Features}
    \begin{enumerate}
        \item \textbf{Improved Map Interaction:}
        Several users struggled with map navigation. Enhancements should include:
        \begin{itemize}
            \item \textbf{Zoom Buttons:} Add visible zoom controls to the map interface for easier navigation.
            \item \textbf{Search Bar:} Include a search bar to allow users to quickly locate specific areas or addresses.
            \item \textbf{Radius Adjustment Slider:} Replace manual radius adjustments with an interactive slider.
        \end{itemize}
        \item \textbf{Additional Map Layers:}
        Users highlighted the need for diverse map layers to support various use cases:
        \begin{itemize}
            \item \textbf{Satellite View:} Offer an aerial view for detailed spatial analysis.
            \item Transport Overlays: Include layers for bus stops, tram stations, and train lines.
            \item \textbf{Infrastructure Layers:} Add overlays for utilities such as sewage systems, water grids, and power lines.
        \end{itemize}
        \item \textbf{Data Export and Sharing:}
        Enable professionals to export data and share insights:
        \begin{itemize}
            \item Add an export feature to save maps and data visualizations as PNG or PDF files.
            \item Introduce social sharing options to disseminate information quickly.
        \end{itemize}
    \end{enumerate}
    \item \textbf{Accessibility and Inclusivity Improvements}
    \begin{enumerate}
        \item \textbf{Enhanced Onboarding}
        Users found the tutorial challenging to follow. Improvements should include:
        \begin{itemize}
            \item \textbf{Step-by-Step Instructions:} Improve \textbf{contrast ratios} for better visibility, especially for users with visual impairments.
            \item \textbf{Scenario-Based Tutorials:} Optimize fonts and UI elements for users with dyslexia or other reading difficulties.
            \item \textbf{Tooltips and Help Icons:} Ensure the \textbf{profile icon} is more prominent and distinguishable from other map elements.
        \end{itemize}
        \item \textbf{UI Design Tweaks}
        To ensure inclusivity:
        \begin{itemize}
            \item \textbf{Step-by-Step Instructions:} Split onboarding into manageable steps, highlighting key features interactively.
            \item \textbf{Scenario-Based Tutorials:} Provide context-specific scenarios to guide users based on their professional needs.
            \item \textbf{Tooltips and Help Icons:} Place contextual help options throughout the interface for quick user assistance.
        \end{itemize}
    \end{enumerate}
    
    \item \textbf{Feature Expansion}
    \begin{enumerate}
        \item \textbf{New Amenity Categories}
        Users found the tutorial challenging to follow. Improvements should include:
        \begin{itemize}
            \item \textbf{Step-by-Step Instructions:} Split onboarding into manageable steps, highlighting key features interactively.
            \item \textbf{Scenario-Based Tutorials:} Provide context-specific scenarios to guide users based on their professional needs.
            \item \textbf{Tooltips and Help Icons:} Place contextual help options throughout the interface for quick user assistance.
        \end{itemize}
    \end{enumerate}

\end{enumerate}