\subsubsection{Frontend Architecture and Implementation}
The frontend of the Magpie project is central to delivering a seamless, responsive, and user-friendly interface for urban planners and casual users. With a focus on providing aggregated data on public amenities, the frontend not only caters to analytical needs but also ensures inclusivity, accessibility, and scalability. This section delves into the technologies, architectural decisions, and user-centric features that make Magpie’s frontend robust and future-ready.

\begin{enumerate}
    \item \textbf{Core Technologies and Frameworks:}
    \begin{enumerate}
        \item \textbf{React and Next.js:} \\
        The combination of React and Next.js is pivotal to the frontend's architecture. React's component-driven approach enables reusable, modular UI elements, fostering rapid development. Next.js enhances this with features like server-side rendering (SSR) and static site generation (SSG), which significantly improve page load times and SEO.\\
        \linebreak
        \textbf{Key benefits: }
        \begin{itemize}
            \item \textbf{Component Reusability:} Components such as data visualization widgets, navigation bars, and filter panels can be reused across multiple pages.
            \item \textbf{SEO Optimization:} With SSR, content is preloaded on the server, making the application more search-engine-friendly.
            \item \textbf{Routing:} Next.js simplifies dynamic and nested routing, accommodating the needs of a complex application.
        \end{itemize}
    \end{enumerate}
    % Key Benefits:
    % Component Reusability: Components such as data visualization widgets, navigation bars, and filter panels can be reused across multiple pages.
    % SEO Optimization: With SSR, content is preloaded on the server, making the application more search-engine-friendly.
    % Routing: Next.js simplifies dynamic and nested routing, accommodating the needs of a complex application.
    

\end{enumerate}