\subsubsection{Frontend Architecture and Implementation}
In the development of this project, we adopted a modern web development stack that incorporated a combination of cutting-edge technologies aimed at optimizing performance, scalability, and maintainability. The integration of these technologies facilitated the creation of a highly efficient and user-friendly web application. This section provides a comprehensive overview of the key technologies used in the frontend of the application, outlining their individual roles and how they collectively contributed to the success of the project.
\begin{enumerate}
    \item \textbf{React} \\
     React is a widely used and powerful JavaScript/TypeScript library that is primarily designed for building user interfaces. It was chosen for this project due to its ability to facilitate the development of dynamic and responsive web applications. In conjunction with \textbf{Next.js}, React allowed for the implementation of a component-driven architecture. This architectural approach enables the reuse of individual UI components throughout the application, which not only streamlined the development process but also ensured greater maintainability in the long term. By breaking down the user interface into smaller, reusable components, we were able to manage and scale the application more effectively. React’s virtual DOM mechanism also ensured efficient rendering of changes to the UI, contributing to an enhanced user experience by minimizing the amount of time required to update the interface.
    \newline
    Moreover, React’s active community and vast ecosystem of libraries and tools made it easier to integrate other technologies and features into the application, further improving its overall functionality and performance. As a result, React played a crucial role in enabling a seamless, interactive experience for users, while also enhancing the maintainability and scalability of the codebase.

    \item \textbf{Next.js} \\
    Next.js is a robust React framework that extends the capabilities of React by offering features such as server-side rendering (SSR) and static site generation (SSG). These features proved to be essential for optimizing the application's load times, which was critical to the project’s success. Server-side rendering allows the server to generate the HTML for the page on each request, as opposed to client-side rendering where the browser must first download and execute JavaScript to generate the page. This process significantly reduces the time required for the page to be displayed to the user, resulting in a faster and more efficient application.

    Additionally, Next.js’ static site generation capabilities allowed us to pre-render pages at build time, meaning that the application could serve static HTML files for commonly accessed pages, further enhancing performance. The combination of SSR and SSG enabled a smooth and responsive user experience.

    Next.js also provided a robust routing system that made it easier to manage the application’s navigation. The framework’s file-based routing system allowed for simple, declarative routing, which contributed to the overall maintainability of the project. Next.js’ out-of-the-box support for features like code splitting and automatic optimization ensured that the application remained highly performant, even as it grew in complexity. \\

    \item \textbf{TailwindCSS} \\
    For styling the application, \textbf{TailwindCSS}, a utility-first CSS framework, was chosen for its ability to enable rapid UI development. Unlike traditional CSS frameworks, which rely on pre-defined UI components and styles, TailwindCSS provides low-level utility classes that can be combined to build custom designs. This approach allowed for greater flexibility in styling the application, as developers were able to apply precise and granular control over the UI without the need for writing custom CSS from scratch.

    One of the key advantages of using TailwindCSS was the speed with which we could implement and modify styles. By utilizing utility classes, we were able to quickly experiment with different design configurations and make adjustments on the fly. This approach also minimized the need for additional CSS files, reducing the overall complexity of the styling process.

    In addition to TailwindCSS, we incorporated several tools to enhance the framework’s capabilities. \textbf{Tailwind-merge} was used to intelligently combine conflicting utility classes, ensuring that styles were applied consistently throughout the application. \textbf{TailwindCSS-animate} provided utility classes for adding animations, which helped bring dynamic visual elements to life without needing to write custom animation logic. These tools, when used in tandem, allowed for more advanced styling and complex animations while maintaining the simplicity and ease of use that TailwindCSS is known for.

    \item \textbf{Mapbox GL and React Map GL} \\
    The integration of interactive maps and advanced visualization features played a key role in the functionality of the project. For this purpose, we utilized \textbf{Mapbox GL} and \textbf{React Map GL}. \textbf{Mapbox GL} is a powerful mapping library that provides advanced capabilities for rendering interactive maps with smooth transitions and animations. It is built to handle large datasets, enabling the display of dynamic geospatial information with high performance and precision. Mapbox GL was used to power the core mapping functionality of the application, allowing users to interact with maps seamlessly.

    To integrate Mapbox GL with our React application, we used \textbf{React Map GL}, a library that provides a simple wrapper around Mapbox GL, making it easier to embed and control interactive maps within React components. React Map GL allowed us to leverage the full potential of Mapbox GL while maintaining a consistent and efficient React-based architecture.

    In addition to basic mapping functionality, we integrated \textbf{Deck.gl} for advanced data visualization. Deck.gl is a framework that enhances Mapbox GL by enabling the display of complex data visualizations on top of interactive maps. This was particularly beneficial for our project, as it allowed us to visualize large datasets in an efficient and interactive manner, providing users with a rich and engaging experience.

    \item \textbf{ShadCN/UI} \\
    To streamline the development of UI components, we incorporated \textbf{ShadCN}, a component library built on top of \textbf{Radix Primitives}. ShadCN combines the flexibility of Radix UI with a more opinionated design system, tailored for projects that require custom styling with the power of TailwindCSS. Radix Primitives offers a set of low-level UI components that are highly customizable, accessible, and composable, making them ideal for building complex user interfaces.

    By using ShadCN, we were able to access a suite of pre-built, customizable components that adhered to accessibility standards, ensuring that our application was usable by all users, including those with disabilities. ShadCN's integration with Radix Primitives allowed us to quickly implement accessible and customizable components, reducing the need for additional development and ensuring consistency across the application.

    The combination of ShadCN’s flexible components and TailwindCSS allowed us to build a visually appealing and highly functional user interface with minimal configuration. Furthermore, ShadCN's tight integration with accessibility guidelines ensured that the application met the required standards for usability, providing an inclusive experience for all users.

    \item \textbf{Radix Primitives} \\
    Radix Primitives played a crucial role in the design and development of our user interface components. Radix Primitives provides a set of foundational, high-quality UI components that serve as the core building blocks for any application. These components are unstyled and highly customizable, giving developers full control over the look and behavior of the elements while ensuring accessibility and functionality out of the box.

    The core of Radix Primitives includes essential UI elements such as buttons, modals, popovers, sliders, tabs, and other interactive components. These primitives are designed to be flexible and composable, allowing developers to combine them in various ways to create complex UI patterns. The components are built to be accessible, ensuring that the user interface remains usable for all users, including those with disabilities, without requiring additional development effort to meet accessibility standards.

    One of the standout features of Radix Primitives is its focus on accessibility. The components are designed with built-in features like keyboard navigation, screen reader support, and proper focus management. This allows the application to be fully functional and accessible across a wide range of devices and assistive technologies, without the need for custom implementation of these features. As accessibility is a fundamental aspect of modern web development, Radix Primitives ensured that the project adhered to the highest standards for inclusive design.

    In addition to accessibility, Radix Primitives allows for extensive customization. While the components come with sensible defaults, they are unstyled, which provides the flexibility to apply custom styles that match the overall design language of the application. This level of customization allowed the development team to ensure that the UI components blended seamlessly with the rest of the application’s design. By combining Radix Primitives with **TailwindCSS**, we could easily customize the components using utility-first CSS classes, speeding up the styling process and ensuring consistency across the entire application.

    The composability of Radix Primitives also played a significant role in the project's development. The library is designed to allow developers to create complex user interface patterns by combining simple components in flexible ways. For instance, we could easily integrate modal dialogs, dropdown menus, and tooltips to form cohesive UI patterns that enhanced the user experience. This approach not only saved development time but also reduced the likelihood of introducing bugs, as we could rely on well-tested, standardized components rather than building these features from scratch.

    By leveraging Radix Primitives, we were able to focus on creating a polished and accessible user interface while avoiding the complexity of building low-level components. The ability to use these pre-built, flexible primitives allowed us to deliver a high-quality user experience with minimal overhead. Radix Primitives was an essential tool in ensuring that the application was both user-friendly and accessible, all while maintaining the flexibility to customize and extend the components as needed for the project.


\end{enumerate}

The integration of these technologies into the frontend of the project provided a solid foundation for building a high-performance, scalable, and user-friendly web application. From the server-side rendering capabilities of Next.js to the highly customizable components offered by ShadCN, each technology played a critical role in the overall success of the project. The use of React and Next.js allowed for a modular and efficient development process, while TailwindCSS and its associated tools streamlined the styling process and enabled rapid UI development. The incorporation of Mapbox GL and Deck.gl brought advanced mapping and data visualization capabilities to the application, making it more interactive and engaging for users. Together, these technologies enabled us to deliver an application that met high standards of code quality, user experience, and performance, ensuring its success in achieving the project’s objectives.

\subsubsection{Client State Management}
    \paragraph{Introduction}
    Effective client state management is a crucial aspect of the \textbf{Magpie} frontend, as it ensures a seamless and consistent user experience. Given the dynamic nature of the application, with interactive elements such as maps, search filters, and property details, it is essential to maintain synchronization between different UI components while preventing errors and inconsistencies. Managing state efficiently is particularly important when handling large datasets that need to be presented in a user-friendly manner across various parts of the application.
    \newline
    \paragraph{React Context}
    To achieve this, the application employs state management techniques that allow for centralized control of the application’s state. One of the core tools used for this purpose is \textbf{React Context}, which provides a simple way to share state across multiple components without having to pass data manually through props. React Context enables the application to manage the user’s preferences, search filters, and saved properties in a global state, ensuring that these settings are accessible across various parts of the application. This method of state management is particularly useful for dynamic features, such as filtering search results based on user input or retaining user selections when navigating between different pages or sections of the platform.
    \newline
    \paragraph{Redux}
    In addition to React Context, the use of more advanced state management libraries such as Redux can be considered for larger-scale applications that require more complex state handling. Redux facilitates a unidirectional data flow and provides a centralized store that allows the state to be updated in a predictable manner. With Redux, the application can ensure that all components are working with the most up-to-date information, whether that involves real-time data from maps or user-selected property preferences. Redux also allows for more complex actions, such as asynchronous data fetching, and can be used in conjunction with middleware like Redux Thunk for handling side effects, further enhancing the application’s performance and stability.
    