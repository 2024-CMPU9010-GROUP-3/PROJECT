Targeted users are users who would use Magpie as a tool for their work. We interviewed 3 individuals who fit this criteria.\\ \\
The targeted user test sessions took the following format:
\begin{enumerate}
    \item \textbf{Getting to know:} this first step is designed to introduce ourselves to the participants, inform ourselves of their background and occupation and set the tone for the rest of the session.
    \item \textbf{Introduction of Magpie:} this second step serves to give an overview of Magpie's purpose, features, data displayed and functionalities.
    \item \textbf{Exploring Magpie + Discussion:} this third step will allow the participants to explore the application, ask question and reflect on how Magpie can be used to meet their needs for their work.
    \item \textbf{Satisfaction survey + End of session:} this last step serves to conclude the session by asking the participants to fill in a satisfaction survey to quantify their experience using Magpie.
\end{enumerate}

\subsubsection{User 7 - Bryan Boyle}
Bryan Boyle is a lecturer at the University College Cork in Occupational
Sciences and Therapy. With a doctorate in Computer Science and Occupational
Therapy, they have published several papers related to inclusivity in public
spaces (\cite{bryanboyleplaygroundinclusion2023}), and the role of technology in
the lives of individuals with disabilities
(\cite{bryanboylechildrenautism2022}).

They gave their contact information during the research survey. They are
categorized as a \emph{targeted user} because they utilise amenity data in their
work, and Magpie could potentially be a tool they use. Firstly, we got to know
Bryan Boyle, his professional experience and the amenity data he uses for his
research. His area of research focuses on topics of child development and
inclusivity.
%bryan amenity info
\begin{figure}[h!]
    \centering
    \includegraphics[width=0.9\textwidth]{images/bryan-amenity-info.png}
    \caption{User Evaluation - Bryan Boyle information}
\end{figure}

\noindent Next, we introduced Magpie and started the uncontrolled session. The
approach was to guide Bryan Boyle through the homepage whilst introducing
Magpie, and then let them explore on their own while discussing their thoughts.
They were able to complete most of the general tasks, except for zooming in and
out. This is because they were not familiar with mouse scrolling, the onboarding
was not clear enough and zoom buttons were not present on the map.
%table of Bryan's general tasks
\begin{table}[h!]
    \centering
    \caption{Usability testing Tasks - Bryan}
    \begin{tabular}{|p{0.4\textwidth}|p{0.1\textwidth}|p{0.1\textwidth}|p{0.2\textwidth}|}
        \hline
        \textbf{Task}                 & \textbf{Status} & \textbf{Difficulty} & \textbf{Errors} \\
        \hline
        Load Magpie application       & Complete        & 2                   & N/A             \\
        \hline
        Sign up                       & Complete        & 1                   & N/A             \\
        \hline
        Log in                        & Complete        & 1                   & N/A             \\
        \hline
        Complete tutorial             & Complete        & 1                   & N/A             \\
        \hline
        Place cursor on map           & Pass            & 3                   & Required help   \\
        \hline
        Zoom in and out               & Fail            & 5                   & Required help   \\
        \hline
        Hold map and navigate         & Complete        & 1                   & N/A             \\
        \hline
        Adjust radius big/small       & Complete        & 1                   & N/A             \\
        \hline
        Clear marker \& radius        & Complete        & 2                   & N/A             \\
        \hline
        Deselect all amenities        & Complete        & 2                   & N/A             \\
        \hline
        Select one or more amenities  & Complete        & 1                   & N/A             \\
        \hline
        Find tutorial and exit midway & Complete        & 2                   & N/A             \\
        \hline
        Log out                       & Complete        & 1                   & N/A             \\
        \hline
    \end{tabular}
\end{table}\\
\newpage Here were the main takeaways from Bryan's session:
\begin{itemize}
    \item \textbf{Data: }they mainly looked at amenities related to
    accessibility such as accessible parking, public toilets, water fountains,
    etc\ldots They would like to see more data such as public transportation,
    facilities like schools, hospitals and stations. What really interests them
    is the type and quality of an amenity surrounding the specific location they
    are searching for, necessitating higher-level information to add to the
    tooltips.\\
    \item \textbf{General: }they really enjoyed using the search bar, they found
    it intuitive and quicker to use than placing a marker on the map. They are
    very interested in the project and look forward to seeing the final
    product.\\
    \item \textbf{Behaviour: } Bryan seems to have less than average proficiency
    with technology. Throughout the session, we noticed that first they used
    Edge as a browser, they got lost looking for the tab among the many windows
    they had open, their browser window was not full screen and additionally
    struggled zooming into the map.
\end{itemize}
The survey answers below complement his vocal feedback. \textbf{Overall}, Bryan
found Magpie to be simple and very straightforward, demonstrated by his
\underline{5 out of 5} score for the user interface. He would use it as a tool
for his PhD research, mainly on desktop which further solidifies our user
persona and why we prioritized desktop compatibility before mobile.
%bryan survey response
\begin{figure}[h!]
    \centering
    \includegraphics[width=0.5\textwidth]{images/survey-bryan.png}
    \caption{User Evaluation - UI Score Bryan Boyle}
\end{figure}\\

\newpage
\subsubsection{User 8 - Dr. Sarah Rock}
Professional user Dr. Sarah Rock is a transport and urban planning specialist
with over 15 years of experience working in both the private and public sector.
As chair of the MSc in Urban Regeneration at the TU Dublin School of
Architecture, she pushes for sustainable and innovative solutions to land use
and urbanisation, specialising in accessibility planning and network design.

We recruited them through social media channels to take part in a user testing
session. They mostly use amenity data to understand an existing area that has
plans for upgrade or demolitions, as well as studying transportation routes to
plan new stops and service new areas.
%sarah amenity info
\begin{figure}[h!]
    \centering
    \includegraphics[width=0.8\textwidth]{images/sarah-amenity-info.png}
    \caption{User Evaluation - Dr. Sarah Rock information}
\end{figure}

\noindent Next, Dr. Rock loaded the Magpie application and started exploring,
whilst discussing each and every feature. They were able to complete all general
tasks while skipping a few. One question they asked was very interesting:
\emph{how do we define amenities?} Different professions within
urban/transport/city planning define amenities in different ways, and they found
our way of defining it interesting and slightly different from the way they
would've defined it: \emph{`Amenities are any sort of public facilities that you
can use.'}
%table of Sarah's general tasks
\begin{table}[h!]
    \centering
    \caption{Usability testing Tasks - Dr. Sarah Rock}
    \begin{tabular}{|p{0.3\textwidth}|p{0.1\textwidth}|p{0.1\textwidth}|p{0.2\textwidth}|}
        \hline
        \textbf{Task}                 & \textbf{Status} & \textbf{Difficulty} & \textbf{Errors}                 \\
        \hline
        Load Magpie application       & Complete        & 2                   & N/A                             \\
        \hline
        Sign up                       & Pass            & 2                   & Tried to sign up on log in page \\
        \hline
        Log in                        & Complete        & 1                   & N/A                             \\
        \hline
        Complete tutorial             & Complete        & 2                   & N/A                             \\
        \hline
        Place cursor on map           & Complete        & 1                   & N/A                             \\
        \hline
        Zoom in and out               & Complete        & 1                   & N/A                             \\
        \hline
        Hold map and navigate         & Complete        & 1                   & N/A                             \\
        \hline
        Adjust radius big/small       & Complete        & 1                   & N/A                             \\
        \hline
        Clear marker \& radius        & Skipped         & N/A                 & N/A                             \\
        \hline
        Deselect all amenities        & Skipped         & N/A                 & N/A                             \\
        \hline
        Select one or more amenities  & Complete        & 1                   & N/A                             \\
        \hline
        Find tutorial and exit midway & Skipped         & N/A                 & N/A                             \\
        \hline
        Log out                       & Skipped         & N/A                 & N/A                             \\
        \hline
    \end{tabular}
\end{table}

\newpage Here were the main takeaways from Dr. Rock's session:
\begin{itemize}
    \item \textbf{Data:} the tooltips on each icon is very interesting. Perhaps cross-check the information there and add/remove details. For example, the bike sharing amenity tooltip has name, number and address but the number seems to be an id and not the number of bikes at the bike sharing station.\\
    \item \textbf{Additional features:} for their work, Dr. Rock likes to have a satellite view of the area they are inspecting. They found that when they are zoomed in at the street level, they find it hard to understand the map. Perhaps adding a satellite layer for enhanced visualisation.\\ Also, an export feature would be amazing, with a scale and table of the amenities found on the location selected on the map.\\
    \item \textbf{Miscellaneous:} the car parking amenity has lots of potential because it relates to the use of public space which is very contested and a big issue in urban planning. Magpie displays that information in an easy manner and there is so much potential to expand the parking detection and help fuel research on how much cars take up public space in cities and inefficiencies using public space for parking especially with limited land.
\end{itemize}
\textbf{Overall}, they stated that Magpie is a fascinating and really useful project in its own right.\\
The radius slider is great, the amenity count is very useful, Dr. Rock can see Magpie being used for planning and research not so much casual user. Points to improve would be more amenity data and revise the information on the amenity tooltips. They scored Magpie's UI a \underline{4.6 out of 5}, the tutorial and the filters losing a star for lack of clarity and intuitiveness.
%sarah survey response
\begin{figure}[h!]
    \centering
    \includegraphics[width=0.7\textwidth]{images/survey-sarah.png}
    \caption{User Evaluation - UI Score Dr. Sarah Rock}
\end{figure}\\


\newpage
\subsubsection{User 9 - Odran Reid}
Professional user Odran Reid has over 30 years of experience in the public, private and NGO sector in Europe. With degrees in economics from Trinity and spatial planning from TU Dublin, they specialise in retail planning, socio-economic analysis and community consultation for local development projects. They are currently a lecturer in spatial planning at the TU Dublin School of Architecture.\\ \\
We recruited them through social media channels to take part in a user testing session.\\
They mostly use amenity data for socio-economic research and consulting for community project developments.\\
%odran amenity info
\begin{figure}[h!]
    \centering
    \includegraphics[width=0.8\textwidth]{images/odran-amenity-info.png}
    \caption{User Evaluation - Odran Reid information}
\end{figure}

\noindent Before the session started, Odran Reid told us they qualified their technological proficiency as `very low' and to be patient with him. This was confirmed when he struggled to share his screen at the start of the session, however he was very comfortable navigating the map, zooming in and out and using the features of the dashboard. Here were the main takeaways:
\begin{itemize}
    \item \textbf{Amenities:} some minor content changes related to the labelling of amenities, for example `library' should be `public library', `public wifi' should be `free public wifi' etc\ldots Also, it would be beneficial to differentiate the types of bike sharing system (Dublin Bikes, Bleeper, Moby Bikes).\\ More data should be added like schools, public transportation, zoning information, vacant buildings and population data to calculate amenity per capita for example.\\
    \item \textbf{Features:} giving the user the chance to draw their own radius could be very interesting. Also, adding a `300 meters' preset for the radius slider would cater to planners more because of the 15 minute walkable rule.\\ This rule defines that 300m is a walkable distance for abled individuals and is used a lot in retail/town planning.\\
    \item \textbf{Miscellaneous:} Odran Reid suggested we take a look at the website \emph{Dublin Smart City} and \emph{POBAL} for more datasets relevant to our project. Improve Magpie with the suggestions above could help with research into evaluating the walkability of Dublin city.
\end{itemize}
\textbf{Overall}, Odran Reid gave some really interesting feedback, different from the other two targeted users. They believe Magpie has commercial potential but needs a lot more datasets and a lot more details to turn this into the go-to tool for planners in any sector. They expressed how they would've enjoyed having this tool earlier on for their consulting work.\\ \\
They scored Magpie's UI a \underline{4.6 out of 5}, the log in and sign up losing a star for lack of email verification and smoothness.
%odran survey response
\begin{figure}[h!]
    \centering
    \includegraphics[width=0.7\textwidth]{images/survey-odran.png}
    \caption{User Evaluation - UI Score Odran Reid}
\end{figure}\\

\newpage
\subsubsection{Targeted users overview}
Interviewing these three users was a really amazing experience. It's easy to put all planners into one basket however, they all have different needs, experiences and wants and these sessions really opened our eyes on that.\\ \\
The main changes these users would like to see are \textbf{more data}, more \textbf{precise information} on the amenity tooltips, an \textbf{export feature} and improvements to the onboarding.\\ \\
The scores they gave on Magpie's UI averaged to \underline{4.8 out of 5}, which is incredible. This score validates Magpie's project vision and end goal: to create a easy to use tool to give a glance view of amenities in an area.
%target survey response summary
\begin{figure}[h!]
    \centering
    \includegraphics[width=0.7\textwidth]{images/survey-target-summary.png}
    \caption{User Evaluation - UI Target Score Average}
\end{figure}\\
