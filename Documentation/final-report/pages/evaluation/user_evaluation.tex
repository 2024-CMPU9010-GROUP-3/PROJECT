\subsubsection{Components of usability}
Based on Nielsen 2012 (https://medium.com/@iizzathisharah/the-five-usability-components-by-jakob-nielsen-detailed-insights-and-examples-90695af5ffb6)

\begin{itemize}
    \item learnability
    \item efficiency
    \item memorabiliy
    \item errors
    \item satisfaction
\end{itemize}

\newpage
\subsubsection{Usability testing}
Magpie has remedied the first challenge of fragmented information on
amenities, and through this user evaluation, we hope to address the second challenge
which is making the access to this information easy, quick \& accessible.\\\\

The goal of the user evaluation is to gain feedback from real users, learn if
Magpie works as expected and assess how user-friendly it is. We will be using 2
main methods to collect both qualitative data through open-ended questions, and
quantitative data from multiple choice questions from which we will derive
insights to improve the Magpie user experience.\\\\

Our approach was as follows:
%user evaluation approach
\begin{enumerate}
    \item Round up users from the market research + seek out others
    \item Conduct online sessions where they would interact with Magpie
    \item Discuss Magpie, explore the features, gather feedback
    \item Fill out satisfaction survey
    \item Synthesize notes from sessions and summarize points to improve
\end{enumerate}

These sessions followed an uncontrolled usability testing approach where we let the users freely roam the application while observing their behaviour interacting with each element, guide them when they were stuck and initiate discussion on the topic of amenities, use cases for the application and feedback on features to improve.\\\\
We used a table with a list of general tasks that the user to quantitatively evaluate each feature on its ease of completion, intuitiveness and action. The list of general tasks increased as the test sessions went on because we were iteratively adding new features.\\\\
The difficulty of the task is related to if they were able to complete it and how much they struggled during it. The status of a task can either be "complete", "pass", or "fail" where "complete" is attributed when the user does the task on their own, "pass" is attributed when the user was able to complete the task but with our help, and "fail" when the user were not able to do the task even with our help. \\\\

\newpage
\subsubsection{User 1 - Paul}
Our first test session was with Paul, a student in technological post-graduate degree. We initially wanted this to be a controlled test session by giving him specific tasks, but found that challenging as he intuitively went on to explore the application on his own.
%table of Paul's general tasks
\begin{table}[h!]
    \centering
    \caption{Usability testing Tasks - Paul}
    \begin{tabular}{|p{0.4\textwidth}|p{0.1\textwidth}|p{0.1\textwidth}|p{0.1\textwidth}|p{0.1\textwidth}|}
        \hline
        \textbf{Task}                                 & \textbf{Status}        & \textbf{Time taken} & \text{Difficulty} & \textbf{Errors}             \\
        \hline
        Load Magpie application                       & Complete               & 20s                 & 1                 & N/A                         \\
        \hline                                        & \textbf{Question type}                                                                         \\
        Sign up                                       & Complete               & 42s                 & 1                 & N/A                         \\
        \hline
        Complete tutorial                             & Complete               & 60s                 & 1                 & N/A                         \\
        \hline
        Place cursor on map and adjust radius to 250m & Fail                   & N/A                 & N/A               & Skipped task                \\
        \hline
        Zoom in to road name level                    & Complete               & 5s                  & 1                 & N/A                         \\
        \hline
        Place cursor on another area                  & Complete               & 5s                  & 1                 & N/A                         \\
        \hline
        Zoom out to see full radius                   & Fail                   & N/A                 & N/A               & Skipped task                \\
        \hline
        Filter to only view "Parking meter" data      & Pass                   & 120s                & 3                 & Required our help           \\
        \hline
        Filter to toggle off all amenities            & Pass                   & 37s                 & 3                 & Required our help           \\
        \hline
        Go through tutorial and exit at Step 3        & Pass                   & 30s                 & 3                 & Couldn't find tutorial icon \\
        \hline
        Log out                                       & Complete               & 20s                 & 2                 & N/A                         \\
        \hline
    \end{tabular}
\end{table}

\newpage
\subsubsection{User 2 - Livia}
Casula user Livia

\newpage
\subsubsection{User 3 - Ben}
Casual user Ben

\newpage
\subsubsection{User 4 - Jakub}
Casual user Jakub

\newpage
\subsubsection{User 5 - Brendan}
Casual user Brendan

\newpage
\subsubsection{User 6 - Anonymous}
Casual user Maira

\newpage
\subsubsection{User 7 - Bryan Boyle}
Professional user Bryan

\newpage
\subsubsection{User 8 - Anonymous}
Professional user Sarah

\newpage
\subsubsection{User 9 - Anonymous}
Professional user Odran
