\subsubsection{User 1 - Brendan}
Brendan is a professor with technological background. They are considered a
casual user.
An uncontrolled test session was conducted where they tested each feature of
Magpie to find faults, system failures and bugs.

\textbf{Main takeaways from Brendan's session: }Magpie has potential for use by certain types of users but needs a lot more functionality. Here is a breakdown of the feedback:
\begin{itemize}
    \item \textbf{Log in/Sign up: } email verification is important, especially
    for a service advertised towards professionals. Also, Brendan did not
    understand the need for username to be created, the username should
    automatically be the email address. As a group, we decided to keep the
    username field as a preference and for future work as the user will be
    greeted by the username chosen when the log in,, or it will display in the
    profile menu.
    \vspace{0.2cm}

    \item \textbf{Tutorial: }the content of step 2 needs to be reworked and use
    more descriptive language. Also, the positioning of certain elements is off
    in Firefox browser. For step 4, the word `dozen' should be changed to
    `several' or similar especially if we are planning on adding more amenity
    data. Lastly, step 5 should point to the map, so that is a bug we need to
    investigate bug.
    \vspace{0.2cm}

    \item \textbf{Dashboard: }we should implement a button to clear the marker
    and all the points from the map, right now the user is forced to reload the
    page to do that. In addition, it could be beneficial to certain users to
    leave the count of toggled off amenities. Also, compress the list of
    amenities to avoid scrolling. We found Firefox to cause a window size
    issue that we have been unable to address.
    \vspace{0.2cm}

    \item \textbf{Map: } plus and minus buttons should be added to zoom in and
    out of the map, especially for users who are not familiar with mouse
    technology. Also, clicking on amenity icons should provide more information.
    Right now, it feels a bit empty. Lastly, some of the selected icons are not
    visually striking, perhaps find a way to emphasize them.
    \vspace{0.2cm}

    \item \textbf{Technical: }when clicking on map, there is a slight offset
    between where the marker appears and where the user clicked. There is cause
    for investigation there. Also, maybe think of implementing something to
    avoid mis-clicking and loosing the original marker position.

    \newpage{}

    \item \textbf{Miscellaneous: }Brendan asked why we were requesting location
    and what we were doing with this data. Accepting Magpie to use the user's
    location only puts down their own location marker on the map indicating
    where they are located. The location information is not processed by the
    Magpie server, it is processed on the user's device. Additionally, we need
    a landing page to present Magpie and put forward the machine learning aspect
    of the project.
\end{itemize}

\textbf{Overall: }the interface is nice, but you might want to focus on
producing a much more data-driven approach for the interface if you want to
attract those kinds of users.

\textbf{Score from survey: }
%brendan survey score
\begin{figure}[h!]
    \centering
    \includegraphics[width=\textwidth]{images/survey-brendan.png}
    \caption{User Evaluation - UI Score Brendan}
\end{figure}

\newpage

\subsubsection{User 2 - Anonymous 1}
This user has a background in technology at the doctorate level. This session
was mostly uncontrolled, they browsed the application, tested the features and
discussed their thoughts with us.

\textbf{Main takeaways from the session: } this user really enjoyed the
presentation of information on the application, they found the amenities easy to
understand and recognizable in the radius of the map. They tried to interact
with the locked elements of the tutorial, and really enjoyed the confetti at the
end of it. Also, before placing a marker on the map because when they accepted
location tracking, their own marker appeared. To make more improvements, they
suggested the following:
\begin{itemize}
    \item \textbf{More features: }They feel like a search bar would help in the
    quest for information in specific location visually unknown to the user.
    They also though the points would appear automatically on the map but
    instead she had to place her own marker. Perhaps there is an issue with
    onboarding not being retained.
    \vspace{0.2cm}

    \item \textbf{Dashboard \& Map: }There is an icon, the water fountain one
    that is neon blue, therefore blends into the white background of the
    dashboard and in the map; probably needs to be changed. Also, more data for
    example on transportation would be a big plus.
    \vspace{0.2cm}

    \item \textbf{Misc: } If you want to appeal to more general users, adding
    more features like location sharing, integrating social interactions and
    make it mobile responsive would be the way to go. Also, a higher level of
    information.
\end{itemize}
\textbf{Overall: }it is a very good application, a very interesting idea to
gather all this information in one place.

\noindent\textbf{Score from survey: }
%maira survey score
\begin{figure}[h!]
    \centering
    \includegraphics[width=0.6\textwidth]{images/survey-maira.png}
    \caption{User Evaluation - UI Score Anonymous 1}
\end{figure}

\newpage
\subsubsection{User 3 - Paul}
Our next session was with Paul, a student in technological undergraduate degree.
They are classified as a general user. They gave their contact email in the
research survey. We initially wanted this to be a controlled test session by
giving him specific tasks, but found that challenging as he intuitively went on
to explore the application on his own.
%table of Paul's general tasks
\begin{table}[h!]
    \centering
    \caption{Usability testing Tasks - Paul}
    \begin{tabular}{|p{0.4\textwidth}|p{0.1\textwidth}|p{0.1\textwidth}|p{0.1\textwidth}|p{0.1\textwidth}|}
        \hline
        \textbf{Task}                                 & \textbf{Status} & \textbf{Time taken} & \textbf{Difficulty} & \textbf{Errors}    \\
        \hline
        Load Magpie application                       & Complete        & 20s                 & 1                   & N/A                \\
        \hline
        Sign up                                       & Complete        & 42s                 & 1                   & N/A                \\
        \hline
        Complete tutorial                             & Complete        & 60s                 & 1                   & N/A                \\
        \hline
        Place cursor on map and adjust radius to 250m & Fail            & Skipped             & Skipped             & Skipped            \\
        \hline
        Zoom in to road name level                    & Complete        & 5s                  & 1                   & N/A                \\
        \hline
        Place cursor on another area                  & Complete        & 5s                  & 1                   & N/A                \\
        \hline
        Zoom out to see full radius                   & Fail            & Skipped             & Skipped             & Skipped            \\
        \hline
        Filter to only view "Parking meter" data      & Pass            & 120s                & 3                   & Required help      \\
        \hline
        Filter to toggle off all amenities            & Pass            & 37s                 & 3                   & Required help      \\
        \hline
        Go through tutorial and exit at Step 3        & Pass            & 30s                 & 3                   & Couldn't find icon \\
        \hline
        Log out                                       & Complete        & 20s                 & 2                   & N/A                \\
        \hline
    \end{tabular}
\end{table}\\
\textbf{Main takeaways from Paul's session: }the map and the amenity data displayed is `excellent', they would find it useful for local areas of the city. One aspect they advise we improve on is to make the choice of amenities more intuitive. This is further supported by they're behaviour trying to click on the icon and amenity title on the dashboard to toggle it on and off, as well as the difficulties they encountered as shown in the general task table.\\
Another point to improve on is to make the profile and tutorial icons more visible, demonstrated by the time it took to find them.\\ \\

\newpage{}

\noindent\textbf{Score from survey: }
%paul survey score
\begin{figure}[h!]
    \centering
    \includegraphics[width=\textwidth]{images/survey-paul.png}
    \caption{User Evaluation - UI Score Paul}
\end{figure}

\newpage{}

\subsubsection{User 4 - Livia}
Our next session was with Livia, another student in a technological undergraduate degree. They are also identified as a casual user who also left their contact in the research survey.\\
This session also started out as a controlled test with a defined set of tasks, but just like Paul, Livia went on to explore the application skipping the tasks.\\
%table of Livias's general tasks
\begin{table}[h!]
    \centering
    \caption{Usability testing Tasks - Livia}
    \begin{tabular}{|p{0.4\textwidth}|p{0.1\textwidth}|p{0.1\textwidth}|p{0.1\textwidth}|p{0.1\textwidth}|}
        \hline
        \textbf{Task}                                 & \textbf{Status} & \textbf{Time taken} & \textbf{Difficulty} & \textbf{Errors} \\
        \hline
        Load Magpie application                       & Complete        & 5s                  & 1                   & N/A             \\
        \hline
        Sign up                                       & Complete        & 16s                 & 1                   & N/A             \\
        \hline
        Complete tutorial                             & Complete        & 44s                 & 1                   & N/A             \\
        \hline
        Place cursor on map and adjust radius to 250m & Complete        & 6s                  & 1                   & N/A             \\
        \hline
        Zoom in to road name level                    & Complete        & 8s                  & 1                   & N/A             \\
        \hline
        Place cursor on another area                  & Fail            & Skipped             & Skipped             & Skipped         \\
        \hline
        Zoom out to see full radius                   & Fail            & Skipped             & Skipped             & Skipped         \\
        \hline
        Filter to only view "Parking meter" data      & Fail            & Skipped             & Skipped             & Skipped         \\
        \hline
        Filter to toggle off all amenities            & Complete        & 18s                 & 1                   & N/A             \\
        \hline
        Go through tutorial and exit at Step 3        & Complete        & 24s                 & 2                   & N/A             \\
        \hline
        Log out                                       & Complete        & 20s                 & 2                   & N/A             \\
        \hline
    \end{tabular}
\end{table}\\
\noindent\textbf{Main takeaways from Livia's session: }very interesting project, useful and great; overall a very clear website. Biggest point of discontent for Livia was the tutorial, they let us know they has dyslexia and the tutorial could've been worded more effectively to cater to them and others with learning/visual impediments. In addition, they tried to interact with the locked elements during the tutorial, suggesting intuition to put in practice what they are reading to validate the information absorbed.\\
They also suggested adding more amenities such as public transports stops, scooter stands and student hubs. They liked how it was easier to understand the information visually compared to Google maps or Apple maps.\\ \\

\newpage{}

\textbf{Score from survey: }
%livia survey score
\begin{figure}[h!]
    \centering
    \includegraphics[width=\textwidth]{images/survey-livia.png}
    \caption{User Evaluation - UI Score Livia}
\end{figure}

\newpage
\subsubsection{User 5 - Ben}
Our next test session was with Ben, another student in a technological post-graduate degree. They are also identified as a casual user who also left their contact in the research survey.\\ \\
Starting this session, we took a more uncontrolled approach and let the users free roam the application without giving them specific tasks to complete. We guided them in the beginning and initiated certain discussions but overall let the users take the reign and think aloud during their exploration process.\\
%table of Ben's general tasks
\begin{table}[h!]
    \centering
    \caption{Usability testing Tasks - Ben}
    \begin{tabular}{|p{0.4\textwidth}|p{0.1\textwidth}|p{0.1\textwidth}|p{0.1\textwidth}|p{0.1\textwidth}|}
        \hline
        \textbf{Task}                 & \textbf{Status} & \textbf{Difficulty} & \textbf{Errors} \\
        \hline
        Load Magpie application       & Complete        & 1                   & N/A             \\
        \hline
        Sign up                       & Complete        & 1                   & N/A             \\
        \hline
        Log in                        & Complete        & 1                   & N/A             \\
        \hline
        Complete tutorial             & Complete        & 1                   & N/A             \\
        \hline
        Place cursor on map           & Complete        & 1                   & N/A             \\
        \hline
        Zoom in and out               & Complete        & 1                   & N/A             \\
        \hline
        Hold map and navigate         & Complete        & 1                   & N/A             \\
        \hline
        Adjust radius big/small       & Complete        & 1                   & N/A             \\
        \hline
        Clear marker \& radius        & Skipped         & Skipped             & Skipped         \\
        \hline
        Deselect all amenities        & Complete        & 1                   & N/A             \\
        \hline
        Select one or more amenities  & Complete        & 1                   & N/A             \\
        \hline
        Find tutorial and exit midway & Skipped         & Skipped             & Skipped         \\
        \hline
        Log out                       & Complete        & 1                   & N/A             \\
        \hline
    \end{tabular}
\end{table}\\
\noindent\textbf{Main takeaways from Ben's session: }the application does exactly what we described it to do- a GIS application to give a at a glance of amenities in Dublin. The overall impression is that it's a very helpful application, easy to use and effective.\\
Points to improve are the loading times for the amenity points, perhaps  directly being logged in after sign up to avoid repetitive steps, make the profile and tutorial icons more visible as they blend into the map, remove mac keyboard icons from the profile bubble, and if possible add more information on each amenity perhaps with tooltips, or add more amenity data like public transportation.\\ \\

\newpage{}

\textbf{Score from survey: }
%ben survey score
\begin{figure}[h!]
    \centering
    \includegraphics[width=\textwidth]{images/survey-ben.png}
    \caption{User Evaluation - UI Score Ben}
\end{figure}

\newpage
\subsubsection{User 6 - Jakub}
Jakub is a professional with a construction and technological background. They were recruited towards the end of product development to test Magpie. They are considered a casual user.\\ \\
This was an uncontrolled test session where Jakub discovered the application on their own, discussing each feature, testing each feature, and were then given a small scenario `Put yourself in the shoes of an urban planner\ldots' to obtain a different kind of feedback from previous sessions.\\
%table of Jakub's general tasks
\begin{table}[h!]
    \centering
    \caption{Usability testing Tasks - Jakub}
    \begin{tabular}{|p{0.4\textwidth}|p{0.1\textwidth}|p{0.1\textwidth}|p{0.1\textwidth}|p{0.1\textwidth}|}
        \hline
        \textbf{Task}                 & \textbf{Status} & \textbf{Difficulty} & \textbf{Errors}    \\
        \hline
        Load Magpie application       & Complete        & 1                   & N/A                \\
        \hline
        Sign up                       & Complete        & 1                   & N/A                \\
        \hline
        Log in                        & Complete        & 1                   & N/A                \\
        \hline
        Complete tutorial             & Complete        & 1                   & N/A                \\
        \hline
        Place cursor on map           & Complete        & 1                   & N/A                \\
        \hline
        Zoom in and out               & Complete        & 1                   & N/A                \\
        \hline
        Hold map and navigate         & Complete        & 1                   & N/A                \\
        \hline
        Adjust radius big/small       & Complete        & 1                   & N/A                \\
        \hline
        Clear marker \& radius        & Pass            & 3                   & Did not see button \\
        \hline
        Deselect all amenities        & Complete        & 1                   & N/A                \\
        \hline
        Select one or more amenities  & Complete        & 1                   & N/A                \\
        \hline
        Find tutorial and exit midway & Skipped         & Skipped             & Skipped            \\
        \hline
        Log out                       & Complete        & 1                   & N/A                \\
        \hline
    \end{tabular}
\end{table}
\noindent\textbf{Main takeaways from Jakub's session: }this tool simplifies the
search for amenities however there are certain items that need to be considered
to improve the application:

\begin{itemize}
    \item \textbf{Log in/Sign up: }Email verification should be included, so
    that the user can confirm they have successfully signed up and also for
    security purposes.
    \vspace{0.2cm}

    \item \textbf{Map: }some of the amenity data doesn't have accurate
    locations, for example public toilets seem to be off by longitude, and
    multi-storey parking data seems incomplete. Also, water fountains are very
    hard to find on the map, we should consider changing its colour. Same with
    the profile and tutorial icon, they are hard to spot on the map. They would
    also like to double click on the map to clear it, more intuitive for them.
    And last thing, amenities with small count are hard to find in the radius,
    maybe make them more visible somehow.
    \vspace{0.2cm}

    \item \textbf{History feature: }doesn't see the use for a casual user, and
    again same for this tool, doesn't see the use for them as a casual user but
    could be useful for a target user.
    \vspace{0.2cm}

    \item \textbf{Extra features: }Search functionality would be very useful for
    those that are lookin for a spot but don't know where it is located
    visually. Also, an export feature would be useful for the scenario of urban
    planning, if I'm to put a report together, a visual from this tool would be
    helpful for illustration.
\end{itemize}
A notable behaviour indicator from them was that they tried to interact with the
elements during onboarding, as have previous users. Due to technical
limitations, we have been unable to make that happen. Future work.
Overall, the tooltips for the icons is very interesting, a suggestion would be
to add the type of parking, zoning information and tariff to the car parking
amenity.

\textbf{Score from survey: }
%jakub survey score
\begin{figure}[h!]
    \centering
    \includegraphics[width=\textwidth]{images/survey-jakub.png}
    \caption{User Evaluation - UI Score Jakub}
\end{figure}

\subsubsection{General user overview}
General users provided very valuable feedback throughout the usability testing
phase of the project. New features have been implemented, reviewed and removed
thanks to these sessions.
For example, zoom buttons were added, tooltips were implemented and icons were
changed. Most notable points raised during these sessions have been the want for
more high level information on the amenities, more amenity data, the visual
feedback from the system when it is loading the points and the onboarding.

The average success rate and average difficulty of a task has been computed in
the table below. The average success rate has been calculated based on the
values of task success, \emph{complete} = 1, \emph{pass} = 0.5 or \emph{fail} =
0.\\
The average difficulty has been calculated based on the values of task
difficulty from 1 = Easy to 5 = Difficult.
\begin{table}[h!]
    \centering
    \caption{Usability testing Tasks - General users Summary}
    \begin{tabular}{|p{0.4\textwidth}|p{0.2\textwidth}|p{0.2\textwidth}|}
        \hline
        \textbf{Task}                 & \textbf{Task Success Rate} & \textbf{Average Difficulty} \\
        \hline
        Load Magpie application       & 100\%                      & 1                           \\
        \hline
        Sign up                       & 100\%                      & 1                           \\
        \hline
        Log in                        & 100\%                      & 1                           \\
        \hline
        Complete tutorial             & 100\%                      & 1                           \\
        \hline
        Place cursor on map           & 100\%                      & 1                           \\
        \hline
        Zoom in and out               & 100\%                      & 1                           \\
        \hline
        Hold map and navigate         & 100\%                      & 1                           \\
        \hline
        Adjust radius big/small       & 100\%                      & 1                           \\
        \hline
        Clear marker \& radius        & 50\%                       & 3                           \\
        \hline
        Deselect all amenities        & 87.5\%                     & 1.5                         \\
        \hline
        Select one or more amenities  & 87.5\%                     & 1.5                         \\
        \hline
        Find tutorial and exit midway & 75\%                       & 2.5                         \\
        \hline
        Log out                       & 100\%                      & 1.5                         \\
        \hline
    \end{tabular}
\end{table}\\
Users struggled to find the icons in the beginning to log out or go through the
tutorial again, as well as the button to clear the radius and points from the
map. Almost all general users skipped the "Clear marker and radius task",
showing that they either did not notice it or did not have a use for it.
%general survey score summary
\begin{figure}[h!]
    \centering
    \includegraphics[width=0.7\textwidth]{images/survey-casual-summary.png}
    \caption{User Evaluation - UI General Score Average}
\end{figure}\\
To conclude, general users gave Magpie a score of \underline{4.5 out of 5}.\\
There was no notable increase in score as the user evaluation progressed between the users, except in the beginning from the session with Brendan where we scored \underline{3.6 out of 5} to the session with Anonymous 1 where we scored \underline{4.6 out 5}. Both users experienced the same version of Magpie, so the difference in score could come down to one personal preference.