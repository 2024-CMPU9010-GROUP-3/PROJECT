We requested a review of our system from UI/UX professional Andrea Curley. \textbf{INTRODUCE ANDREA BACKGROUND}.\\
The goal of this review is to evaluate the user interface of Magpie at different stages of development and rate its user-friendliness in regards to key UI/UX general guidelines.\\
\textbf{UX general guidelines: } outline them here \\ \\

Two session were conducted: one on November 13 after the publication of our first minimum viable product, and the other on \textbf{INSERT DATE} at the end of the development timeline. Both sessions were conducted online through a videoconference meeting on Teams and took the following form:
\begin{enumerate}
    \item Presentation of Magpie
    \item Free-roaming of the application by Professor Curley
    \item Questionnaire
    \item Discussion \& end of review
\end{enumerate}
The questionnaire includes questions on visual design, information architecture, data quality \& integration, technical performance, compliance and overall assessment of Magpie as shown in the table below. Different types of questions were included, such as "Closed" "Scale" and "Open" to allow for both the quantitative and qualitative measure of Magpie. \textbf{SUPPORTING EVIDENCE ON MIXED SURVEY QUUESTIONS}\\ \\
%table show survey questions and type
\begin{table}[h!]
    \centering
    \caption{Expert Review Questionnaire}
    \begin{tabular}{|p{0.03\textwidth}|p{0.82\textwidth}|p{0.15\textwidth}|}
        \hline
            & \textbf{Question}                                                                                                        & \textbf{Question type} \\
        \hline
        Q1  & How user friendly is the log-in/sign up page?                                                                            & Closed                 \\
        \hline
        Q2  & How user-friendly is the on-boarding process                                                                             & Closed                 \\
        \hline
        Q3  & How effective is the visual hierarchy of the information on the dashboard?                                               & Closed                 \\
        \hline
        Q4  & Rate the clarity of the map visualization                                                                                & Closed                 \\
        \hline
        Q5  & How intuitive is the organization of amenity data categories?                                                            & Closed                 \\
        \hline
        Q6  & Rate the following features from Worst (1) to Best (5)                                                                   & Scale                  \\
        \hline
        Q7  & How comprehensive is the amenity data coverage for Dublin city?                                                          & Closed                 \\
        \hline
        Q8  & How valuable do you think this tool would be for the following use cases - 1: Not valuable at all, 5: Extremely valuable & Scale                  \\
        \hline
        Q9  & Any additional comments on why this tool would useful/impractical for the above use cases?                               & Open                   \\
        \hline
        Q10 & Evaluate the following technical aspects from Worst (1) to Best (5)                                                      & Scale                  \\
        \hline
        Q11 & Rate the application's compliance with the items below from Worst (1) to Best (5)                                        & Scale                  \\
        \hline
        Q12 & Any additional comments regarding our application?                                                                       & Open                   \\
        \hline
    \end{tabular}
\end{table}

\subsubsection{Session 1}
The first review provided very valuable insights on Magpie's workflow, user interface and technical components. These were the main takeaways:\\ \\
\textbf{Landing Page: }
Upon loading Magpie, Professor Curley was directly taken to the mapview, which was not supposed to happen. After the review, we investigated the cause of this event and uncovered a bug in the authentication which we have been working on. Following this event, she suggested creating a landing page or some sort of introduction to ease the user into discovering Magpie.\\ \\
\textbf{Onboarding: }
Due to the bug explained above, the onboarding did not automatically start as it should have upon login. Nevertheless, Professor Curley said that the user may want to intuitively press on the elements being highlighted during the tutorial, as she tried to do. This adds to the feedback received during casual testing for the implementation of this feature. Unfortunately, due to a technical limitations we are not able to solve it, only provide make certain changes to dissuade the user of doing so.\\
In addition, Professor Curley suggested there should be an option to exit the tutorial at any time for users who don't want to sit through it. Lastly, the tutorial shoud be more visually striking and engaging in order to leave a lasting impression on the user.\\ \\
\textbf{Dashboard \& Map: }
Currently, the hierarchy of items on the dashboard does not make sense to the average, and is not intuitive to use. All the amenities are displayed when only one is selected (as shown in figure 3.3) and their count displays zero, which the user might interpret as there are zero other amenities in the area in addition to the one I selected.\\
The icons on the map are not visible enough, and zooming in \& out on the map may not be intuitive to the range of users and devices. Adding zoom buttons could help bridge that gap. \\
Currently, Professor Curley noted that there is a disconnect between the map and the dashboard whereas they should be looked as one. She suggested adding amenity icons to the dashboard to help bridge that gap.\\
%figure for the old confusing dashboard
\begin{figure}
    \fbox{\includegraphics[width=\textwidth]{images/old-dashboard_opt.png}}
    \caption{V.0.1 dashboard \& map when 2 amenities are selected}
\end{figure}

\textbf{Filters: }
If there are no amenities found in the radius of search, a message should pop up to tell the user so. Currently, it is not very clear if there are amenities present in the chosen area especially due to the small size \& faded color of the icons. \\ \\
\textbf{Log in/sign up: }
When trying to log in with credentials that don't exist, the system should return a proper error such as "username doesn't exist". Log out and account sign up went smoothly. Professor Curley questioned the benefits of logging for Magpie, to which we stated:
Magpie was conceived with the idea of providing a service to working professionals; therefore logging in will allow the implementation of further features such as safeguarding their previous searches, storing exported reports, connecting with other members of your organization and much more.
%figure for the log in error
\begin{figure}
    \centering
    \fbox{\includegraphics[width=0.4\textwidth]{images/old-login-error.png}}
    \caption{Error when inputting non-existant username and password}
\end{figure}

\textbf{Survey response: }
Below are the answers to the expert review survey. The response to the survey help complement the oral feedback received during the expert review and provide some quantitative data as a baseline for the next evaluation. The two open-ended questions, to which she asked us to elaborate further, will help us review future open-ended questions and ensure they give enough information for the user to answer.A score has been attributed to each section of the survey based on the responses from Andrea.\\\\
The first section covers usability of the main items of Magpie's user interface which scored 3.5 out of 5. The items which brought the score down are the onboarding and the clarity of the map visualization, further supported by the vocal feedback Andrea provided on the un-intuitive flow and disconnect between the map and the dashboard.\\
The second section covers the information architecture and data quality of the amenities. It looks at the features displaying the information and the score reflects the problems aforementioned with the dashboard as well as the incomplete profile menu. Furthermore, Q8 asks Andrea to assume the value of Magpie as a tool for different use cases where our target users (Urban planners and Event planners) were rated as "Valuable". This rating is useful but it remains an assumption.\\
Lastly, the third section covers the technical aspects of Magpie, and the low score of 2.88/5 reflects a major authentication bug encountered during the testing session, as well as lagging of the points on the map due to technical difficulties.\\
Overall, Magpie scored \underline{3.13 out of 5} for this first UX review session. This is the baseline, and the objective for the next session is to score above \underline{3.5 out of 5} overall, and significantly improve the scores for the onboarding, the dashboard flow and the system responsiveness.\\\\
\textbf{Summary: }
To conclude the first session, Professor Curley found our interface sleek and minimalistic. However, she suggested that if we want to remain with this style, we need to ensure there is as little room as possible for ambiguity and confusion. The user needs to find it easy to move from one feature to another and understand the triggers. Currently, Magpie looks so sleek that the user may not be able to see what they want.
%figure of 1st survey responses = UI
\begin{figure}
    \centering
    \fbox{\includegraphics[width=0.7\textwidth]{images/ux-survey1-ui.png}}
    \caption{UX Expert survey response on Magpie UI}
\end{figure}
%figure of 1st survey responses = Data
\begin{figure}
    \centering
    \fbox{\includegraphics[width=0.7\textwidth]{images/ux-survey1-data.png}}
    \caption{UX Expert survey response on Magpie Information Architecture}
\end{figure}
%figure of 1st survey responses = Technical
\begin{figure}
    \centering
    \fbox{\includegraphics[width=0.7\textwidth]{images/ux-survey1-technical.png}}
    \caption{UX Expert survey response on Magpie Technical Performance}
\end{figure}
%figure of 1st survey responses = Summary
\begin{figure}
    \centering
    \fbox{\includegraphics[width=0.7\textwidth]{images/ux-survey1-summary.png}}
    \caption{UX Expert score of Magpie}
\end{figure}

\subsubsection{Session 2}