We requested a review of our system from Accessibility expert Damian Gordon. He served on the board of the National Disability Authority, who advise the Irish Government on all matters related to disability, and he has worked with a wide range of disability organisations, include the Centre Remedial Clinic, the National Council for the Blind, Arthritis Ireland, and the Aging Well Network. He has contributed to the development of hardware, software, legislation and training related to disability awareness and accessibility. \\\\
The goal of this review is to evaluate the accessibility of the Magpie user interface and evaluate it with regards to key accessibility general guidelines.
\begin{enumerate}
    \item Accessibility: visual-auditory-cognitive-physical impairments
    \item Inclusiveness: people for who English isn't their first language, older people, gender and affective issues
    \item Universal design: an app for everyone, for different devices, different browsers and networks
\end{enumerate}
\fcolorbox{yellow}{yellow}{Accessibility general guidelines: outline them here}\\\\
The session was conducted on November 21st during the usability testing phase of Magpie. The session was conducted online through videoconference meeting on Teams and took the following format:
\begin{enumerate}
    \item Presentation of Magpie
    \item Scenario given to use the application
    \item Discussion
    \item Questionnaire \& end of review
\end{enumerate}
The scenario given can be seen in the table below. A scenario was given so as to simulate the use of Magpie by one of our target users through the lense of an accesibility expert.
%table for scenario
\begin{table}[h!]
    \centering
    \caption{Scenario for the Accessibility review}
    \begin{tabular}{|p{\textwidth}|}
        \hline
        \textbf{Scenario:}                                                                                                                                                                                                                                                                                         \\
        \hline
        You are an architect contracted by Dublin City Council to expand the Dominick Street Recreation Centre, located on Dominick Street Lower, Dublin 1. As part of your assignment, you need to plan the expansion in a way that integrates effectively with the surrounding community and existing amenities. \\
        \hline
        \textbf{Scenario tasks:}                                                                                                                                                                                                                                                                                   \\
        \hline
        \begin{enumerate}
            \item \textbf{Locate the Community Centre:} Use the GIS application to locate the Dominick Street Recreation Centre on the map.
            \item \textbf{Identify nearby amenities:} Select the public amenities you think are relevant within a 500-meter radius of the recreation centre. These can include but are not limited to: bicycle stands, parking spaces, public wi-fi spots, public toilets.
            \item \textbf{Analyse amenity density:} Based on your findings, determine which types of amenities are abundant and which are lacking around the centre.
            \item \textbf{Assess accesibility:} Check how accessible the recreation centre is by identifying nearby transportation options. Note any gaps in accessibility that might need addressing.
            \item \textbf{Plan for additional amenities:} Suggest which new amenities should be added as part of the recreation centre's expansion to better serve the community. For example: if car parking is insufficient, recommend additional parking spaces.
        \end{enumerate}                                              \\
        \hline
    \end{tabular}
\end{table}
\\
The questionnaire is the same as the one given to the users who participated in the usability testing. In hindsight, a tailored questionnaire should've been created for Damian Gordon, to specifically evaluate key accessibility items in Magpie.
\fcolorbox{yellow}{yellow}{YOU CAN STILL CREATE IT NO???}\\\\

Throughout the session, we were also observing Damian Gordon's behaviour and if he was able to complete general tasks related to the main features of Magpie. The results can be seen in the table below, it recorded if they were able to complete the task, how difficult it was for them based on behavioural cues and any errors bugs encountered during the task. We were not able to record the time taken for each task due to the informal administration of them, the user was not aware we were "grading" them in a sense.\\
The difficulty of the task is related to if they were able to complete it and how much they struggled during it. The status of a task can either be "complete", "pass", or "fail" where "complete" is attributed when the user does the task on their own, "pass" is attributed when the user was able to complete the task but with our help, and "fail" when the user were not able to do the task even with our help. \\
%table for general tasks completion & difficulty
\begin{table}[h!]
    \centering
    \caption{General tasks score for the Accessibility review}
    \begin{tabular}{|p{0.3\textwidth}|p{0.1\textwidth}|p{0.1\textwidth}|p{0.3\textwidth}|}
        \hline
        \textbf{General task:}           & \textbf{Status} & \textbf{Difficulty} & \textbf{Errors}                                                                   \\
        \hline
        Load Magpie application          & Complete        & 1                   &                                                                                   \\
        \hline
        Sign up new account              & Complete        & 1                   &                                                                                   \\
        \hline
        Log in                           & Complete        & 1                   &                                                                                   \\
        \hline
        Go through onboarding            & Pass            & 2                   & Technical issues with onboarding overlay + user tried to click on locked elements \\
        \hline
        Place cursor on map              & Complete        & 1                   &                                                                                   \\
        \hline
        Zoom in and out                  & Fail            & 5                   & Did not know how to use the mouse + onboarding explanation confusing              \\
        \hline
        Hold map and navigate            & Complete        & 2                   &                                                                                   \\
        \hline
        Adjust radius big/small          & Complete        & 1                   &                                                                                   \\
        \hline
        Clear marker and radius from map & N/A             & N/A                 & Feature not used                                                                  \\
        \hline
        Deselect all amenities           & Complete        & 1                   &                                                                                   \\
        \hline
        Choose certain amenities         & Complete        & 1                   &                                                                                   \\
        \hline
        Find onboarding and exit midway  & Pass            & 4                   & Could not find onboarding button                                                  \\
        \hline
        Logout                           & Pass            & 3                   & Could not find profile button                                                     \\
        \hline
    \end{tabular}
\end{table}

The review provided some useful insights with regards to missing features to complete the scenario and areas of Magpie that comply with accessibility standards. Below were the main takeaways:\\
\textbf{Dashboard: }
Damian Gordon had some trouble locating the recreation centre during the scenario, and advise that have a search bar would remove the difficulty and give options for those who may be visually impaired to find a location on the map, and also for those who may not know where the area they're looking for is located.\\\\
\textbf{Onboarding: }
Damian Gordon found the steps in the onboarding very wordy and too long, which may cause some users to skip through or not retain the information present there. In addition, some of the wording could be improved related to navigating the map \fcolorbox{yellow}{yellow}{as shown in the figure below -- INSERT SCREENSHOT}. This issue was further illustrated by the difficulty Damian Gordon encountered trying to zoom in and out on the map, having not understood how to do it. Addressing the content and flow of the onboarding is important to ensure it uses language easily understandable by all kinds of users.\\\\
\textbf{Map: }
Going back to the zooming feature Damian Gordon struggled with, adding zoom in and zoom out buttons directly on the map would offer an alternative to those not able to use the mouse for the action, further improving Magpie's accessibility.\\In addition, the profile and onboarding icons blended into the map which made it difficult for Damian Gordon to go back to the onboarding or log out.\\\\

Lastly, his survey responses \fcolorbox{yellow}{yellow}{BLA BLA BLA TEXT TEXT BLA...}
