\documentclass{report}
\usepackage{titling} % For subtitle
\usepackage{tocloft} % For customizing TOC formatting
\usepackage{soul} % For highlighting text
\usepackage{xcolor}
\usepackage{graphicx}
\usepackage{tabularx} % for automatic column width adjustment


% Manually setting the margins
\setlength{\textwidth}{\dimexpr 15.4cm}  % Width of text area
\setlength{\textheight}{\dimexpr 23.4cm} % Height of text area
\setlength{\oddsidemargin}{-0.25in}      % Left margin (negative value reduces margin)
\setlength{\evensidemargin}{-0.25in}     % Even page left margin (for two-sided printing)
\setlength{\topmargin}{-0.5in}           % Top margin
\setlength{\headheight}{14pt}            % Space for header (optional)
\setlength{\headsep}{25pt}               % Space between header and text
\setlength{\footskip}{30pt}              % Distance from bottom of text to footer

% Set the main font to Helvetica
\usepackage[scaled]{helvet}
\renewcommand{\rmdefault}{phv}

% Set highlight colour
\sethlcolor{yellow}

\begin{document}

% Title Page Formatting
\includegraphics[width=0.5\textwidth]{Figures/tud_logo.png}
\title{MSc in Computing - Team Project}
%\title{User Evaluation Report - Magpie (Group 3)}
\author{Anais Blenet\\Saul Burgess\\Yuanshuo Du\\Jessica Fornetti\\Andreas Kraus\\Kaustubh Trivedi}
\date{\today}

% Customizing TOC formatting
\renewcommand{\cfttoctitlefont}{\hfill\Huge\bfseries} % Center TOC title
\renewcommand{\cftaftertoctitle}{\hfill}

\maketitle % Generates the title page

% Table of Contents
\tableofcontents
\newpage

% Main Body
\chapter{Introduction}
\section{Proposed Hypothesis}
Our project's goal is to provide a easy-to-use Geographical Information Service
in Dublin City. In our preliminary research, we couldn’t find a singular
system that allowed users to gain a general overview of public amenities, for
example, parking, bike infrastructure, or public transport. In Ireland and the
United Kingdom, the prevalence of proper digitised records in county
administrations varies wildly (Lynn et al., 2023). Some make use of
state-of-the-art geographical information systems (GIS) while others rely on
spreadsheets which are manually kept up to date. (McGuirk and MacLaran, 2001) A
system that would allow users to quickly inspect a combined dataset grounded in
automatically generated, real-world data could accelerate processes like planning permissions, urban development, or resource allocation.
Prior to user evaluation, exploratory work was conducted in the forms of a
market research survey to answer key demographic \& product questions:
\begin{enumerate}
    \item Who is our primary target user?
    \item What kind of amenity data do they access and how?
    \item What devices/tools do they primarily use?
    \item Are they satisfied with those tools?
    \item Would they consider Magpie useful in filling the gaps in their
          toolset?
\end{enumerate}
Responses from the survey further allowed us to confirm our target demographic
(figure 1.1), find out the proportion of users using amenity data for their work
(figure 1.2), what type of amenity data  they require access to (figure 1.3) and
why current tools are unsatisfactory (figure 1.4).
\begin{figure}
    \centering
    \begin{minipage}{0.45\textwidth}
        \centering
        \fbox{\includegraphics[width=\textwidth]{Figures/fig1.png}}
        \caption{Target user sectors}
        \label{fig:plot1}
    \end{minipage}
    \hfill
    \begin{minipage}{0.45\textwidth}
        \centering
        \fbox{\includegraphics[width=\textwidth]{Figures/fig2.png}}
        \caption{Amenity Access distribution}
        \label{fig:plot2}
    \end{minipage}
\end{figure}
These responses also cemented the need of Magpie (figure 1.5) for both casual
users (User A) \& professionals who require amenity data (User B).
\begin{figure}
    \centering
    \begin{minipage}{0.5\textwidth}
        \centering
        \fbox{\includegraphics[width=\textwidth]{Figures/fig3.png}}
        \caption{Current amenity data accessed}
        \label{fig:plot3}
    \end{minipage}
    \hfill
    \begin{minipage}{0.6\textwidth}
        \centering
        \fbox{\includegraphics[width=\textwidth]{Figures/fig4.png}}
        \caption{Current tools \& satisfaction rate}
        \label{fig:plot4}
    \end{minipage}
    \hfill
    \begin{minipage}{0.6\textwidth}
        \centering
        \fbox{\includegraphics[width=\textwidth]{Figures/fig5.png}}
        \caption{Magpie potential}
        \label{fig:plot5}
    \end{minipage}
\end{figure}

\noindent{}The survey also helped us implement additional features (figure
1.6,1.7 \& 1.8) prior to the user evaluation such as a dashboard with search functionality and filters.\\ \\

\begin{figure}
    \centering
    \begin{minipage}{0.45\textwidth}
        \centering
        \fbox{\includegraphics[width=\textwidth]{Figures/fig6.png}}
        \caption{Additional features rating}
        \label{fig:plot6}
    \end{minipage}
    \hfill
    \begin{minipage}{0.45\textwidth}
        \centering
        \fbox{\includegraphics[width=\textwidth]{Figures/fig7.png}}
        \caption{Version 1 of high fidelity prototype}
        \label{fig:plot7}
    \end{minipage}
    \hfill
    \begin{minipage}{0.45\textwidth}
        \centering
        \fbox{\includegraphics[width=\textwidth]{Figures/fig8.png}}
        \caption{Version 2 of high-fidelity prototype}
        \label{fig:plot8}
    \end{minipage}
\end{figure}

\noindent{}Magpie has remedied the first challenge of fragmented information on
amenities, and through this user evaluation, we hope to address the second challenge
which is making the access to this information easy, quick \& accessible.

\chapter{Experimental Methods}
The goal of the user evaluation is to gain feedback from real users, learn if
Magpie works as expected and assess how user-friendly it is. We will be using 2
main methods to collect both qualitative data through open-ended questions, and
quantitative data from multiple choice questions from which we will derive
insights to improve the Magpie user experience.
\section{Usability Testing}
\subsection{Casual Think-Aloud Protocol}
\begin{enumerate}
    \item \underline{Objective:} Obtain quick feedback on frontend features during the development \& implementation process
    \item \underline{Conditions:} Oral feedback, written notes
    \item \underline{Methodology:}
          \begin{enumerate}
              \item Request feedback from users in the immediate circle
              \item Work together with the user through the app and the feature we want to test
              \item Observe workflow of the user, discuss freely on their manner of interaction with the feature
              \item Build on this feedback to make changes or implement the new feature
          \end{enumerate}
    \item \underline{Baseline \& Evaluation metrics:} No baseline; user experience is evaluated "casually", meaning informally \& subjectively at each person's discretion. This method serves as the "Step 0" of usability testing in helping us implement "Draft 1" of new features.
\end{enumerate}
\subsection{Uncontrolled (Remote)}
\begin{enumerate}
    \item \underline{Objective:} Evaluate user experience in an uncontrolled
          environment, assess overall functioning of Magpie \& identify any
          potential usability issues
    \item \underline{Conditions:} Online followed up by survey
    \item \underline{Methodology:}
          \begin{enumerate}
              \item Share the link to Magpie online to wide user-base to request their participation in testing Magpie
              \item Send them survey to rate their experience
              \item Analyse the responses \& present the results
          \end{enumerate}
    \item \underline{Baseline \& Evaluation metrics:} No baseline; user experience will be evaluated through a 10 question satisfaction survey (figure 2.1) to collect quantitative data only. Minimum number of responses is 20 to produce valuable insights.
\end{enumerate}

\begin{figure}
    \begin{minipage}{\textwidth}
        \centering
        \fbox{\includegraphics[width=\textwidth]{Figures/fig9.png}}
        \caption{User Experience survey (Uncontrolled)}
        \label{fig:plot9}
    \end{minipage}
\end{figure}

\subsection{Controlled (Remote)}
\begin{enumerate}
    \item \underline{Objective:} Analyze how users interact with Magpie in a
          controlled environment guided by specific instructions and tasks to complete
          \& identify any potential usability issues
    \item \underline{Conditions:} Online through videoconference tools
          (Zoom/GoogleMeet/Teams)
    \item \underline{Methodology:}
          \begin{enumerate}
              \item Reach out to casual users who left their contact email in the market research survey \& request their participation
              \item Schedule the test session
              \item Conduct the test session
              \item Collect qualitative data from users through open-ended questions pre-during-post test
              \item Collect quantitative data from users through user experience survey post-test
              \item Analyse responses \& present results
          \end{enumerate}
    \item \underline{Baseline \& Evaluation metrics:} No baseline; user
          experience will be evaluated on range of metrics in table 2.1 below and a user experience survey (\hl{figure 2.2 - should i make a different one?})\\
          rate of completion of tasks (success/fail)
          time of completion of tasks (include threshold)
          number of bugs encountered (include threshold)
          difficulty of tasks (observational - oral data)
\end{enumerate}
\begin{table}[h!]
    \centering
    \caption{Metrics for Controlled Usability testing}
    \label{tab:table2}
    \begin{tabularx}{\textwidth}{|p{0.6\textwidth}|X|X|}
        \hline
        \textbf{Metric}   & \textbf{Threshold} \\ \hline
        Task Success Rate & 50\%               \\ \hline
        Time on Task      & Custom             \\ \hline
        Error Rate        & Custom             \\ \hline
    \end{tabularx}
\end{table}
\subsection{Field test (Remote)}
\begin{enumerate}
    \item \underline{Objective:} Analyze how users interact with Magpie in a
          controlled environment guided by their own tasks for their day-to-day work
          \& identify any potential usability issues
    \item \underline{Conditions:} Online through videoconference tools
          (Zoom/GoogleMeet/Teams)
    \item \underline{Methodology:} Reach out to professional users who left their contact
          email in the market research survey, request their participation in user experience survey, schedule the test session, conduct the test session, collect answers from the user experience survey, analysis \& results.
    \item \underline{Baseline \& Evaluation metrics:} No baseline; user
          experience will be evaluated on their feedback through user experience survey and observational analysis (\hl{figure 2.3})
\end{enumerate}
\section{Expert Review}
We have requested an expert review from a UI/UX professional named Andrea Curley. The goal of this review is to evaluate the user interface of Magpie, rate its user-friendliness in regards to UI/UX general guidelines and the application's compliance with the EAA (European Accessibility Act).\\
The expert review will be conducted online through a videoconference meeting on Teams and will take the following form:
\begin{enumerate}
    \item Presentation of Magpie
    \item Free-roaming of the application by Professor Curley
    \item Questionnaire
    \item Discussion \& end of review
\end{enumerate}
The questionnaire will include questions on visual design, information architecture, data quality \& integration, technical performance, compliance and overall assessment of Magpie as shown in Table 2.1.
\begin{table}[h!]
    \centering
    \caption{Expert Review Questionnaire}
    \label{tab:table1}
    \begin{tabularx}{\textwidth}{|p{0.6\textwidth}|X|X|}
        \hline
        \textbf{Question}                                                                                                                                                                                                                & \textbf{Question type} & \textbf{Answer type} \\ \hline
        How user friendly is the log-in/sign up page?                                                                                                                                                                                    & Closed                 & Multiple choice      \\ \hline
        How user-friendly is the on-boarding process                                                                                                                                                                                     & Closed                 & Multiple choice      \\ \hline
        How effective is the visual hierarchy of the information on the dashboard?                                                                                                                                                       & Closed                 & Multiple choice      \\ \hline
        Rate the clarity of the map visualization                                                                                                                                                                                        & Closed                 & Multiple choice      \\ \hline
        How intuitive is the organization of amenity data categories?                                                                                                                                                                    & Closed                 & Multiple choice      \\ \hline
        Rate the following features from Worst (1) to Best (5) = Onboarding, Radius scaling, filter options completeness, profile menu                                                                                                   & Closed                 & Scale                \\ \hline
        How comprehensive is the amenity data coverage for Dublin city?                                                                                                                                                                  & Closed                 & Multiple choice      \\ \hline
        How valuable do you think this tool would be for the following use cases - 1: Not valuable at all, 5: Extremely valuable = Urban planning, Resource allocation, Planning permissions, Event planning, Education, Travel planning & Closed                 & Scale                \\ \hline
        Any additional comments on why this tool would useful/impractical for the above use cases?                                                                                                                                       & Open-ended             & Textbox              \\ \hline
        Evaluate the following technical aspects from Worst (1) to Best (5) = Loading speed, System responsiveness, Data update frequency, Filter functionality, Radius selection                                                        & Closed                 & Scale                \\ \hline
        Rate the application's compliance with the items below from Worst (1) to Best (5) = Accessibility, GIS data standards, GDPR                                                                                                      & Closed                 & Scale                \\ \hline
        Any additional comments regarding our application?                                                                                                                                                                               & Open-ended             & Textbox
    \end{tabularx}
\end{table}

\chapter{Results}
\section{Usability testing}
\subsection{Casual feedback}
\begin{itemize}
    \item Onboarding feature implementation \hl{SAUL}
    \item Dashboard implementation \hl{KAUSTUBH}
    \item Avatar implementation \hl{YUANSHUO}
\end{itemize}
\subsection{Uncontrolled (Remote)}
\subsection{Controlled (Remote)}
\subsection{Field-test (Remote)}
\section{Expert review}


% References & Table of Figures Page
\newpage
\addcontentsline{toc}{chapter}{References}
\chapter*{References}
% Use \bibliography{yourbibfile} with a .bib file if available
\begin{itemize}
    \item Lynn, Theodore et al. (Apr. 2023). “Web Accessibility of Irish Local Government Websites”. In: ICDS 2023 : The
          Seventeenth International Conference on Digital Society.
    \item McGuirk, Pauline M and Andrew MacLaran (2001). “Changing approaches to urban planning in an ‘entrepreneurial
          city’: the case of Dublin”. In: European Planning Studies 9.4, pp. 437–457.
\end{itemize}

\newpage
\addcontentsline{toc}{chapter}{Table of Figures}
\listoffigures
\listoftables

\end{document}
