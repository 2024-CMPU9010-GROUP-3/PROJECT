\documentclass{report}
\usepackage{titling} % For subtitle
\usepackage{tocloft} % For customizing TOC formatting
\usepackage{soul} % For highlighting text
\usepackage{xcolor}
\usepackage{graphicx}


% Manually setting the margins
\setlength{\textwidth}{\dimexpr 15.4cm}  % Width of text area
\setlength{\textheight}{\dimexpr 23.4cm} % Height of text area
\setlength{\oddsidemargin}{-0.25in}      % Left margin (negative value reduces margin)
\setlength{\evensidemargin}{-0.25in}     % Even page left margin (for two-sided printing)
\setlength{\topmargin}{-0.5in}           % Top margin
\setlength{\headheight}{14pt}            % Space for header (optional)
\setlength{\headsep}{25pt}               % Space between header and text
\setlength{\footskip}{30pt}              % Distance from bottom of text to footer

% Set the main font to Helvetica
\usepackage[scaled]{helvet}
\renewcommand{\rmdefault}{phv}

% Set highlight colour
\sethlcolor{yellow}

\begin{document}

% Title Page Formatting
\title{MSc in Computing - Team Project}
\title{User Evaluation Report - Magpie (Group 3)}
\author{Anais Blenet\\Saul Burgess\\Yuanshuo Du\\Jessica Fornetti\\Andreas Kraus}
\date{\today}

% Customizing TOC formatting
\renewcommand{\cfttoctitlefont}{\hfill\Huge\bfseries} % Center TOC title
\renewcommand{\cftaftertoctitle}{\hfill}

\maketitle % Generates the title page

% Table of Contents
\tableofcontents
\newpage

% Main Body
\chapter{Introduction}
\section{Proposed Hypothesis}
Our project goal is to develop an application that provides “services at a glance” in county Dublin. In our preliminary research, we couldn’t find a singular system that allowed users to gain a general overview of public amenities, for example, parking, bike infrastructure, or public transport. In Ireland and the United Kingdom, the prevalence of proper digitised records in county administrations varies wildly (Lynn et al., 2023). Some make use of state-of-the-art geographical information systems (GIS) while others rely on spreadsheets which are manually kept up to date. (McGuirk and MacLaran, 2001) A system that would allow users to quickly inspect a combined dataset grounded in automatically generated, real-world data could accelerate processes like planning permissions, urban development, or resource allocation.\\ \\
Prior to user evaluation, exploratory work was conducted in the forms of a market research survey to answer key demographic \& product questions:
\begin{enumerate}
    \item Who is our primary target user?
    \item What kind of amenity data do they access and how?
    \item What devices/tools do they primarily use?
    \item Are they satisfied with those tools?
    \item Would they consider Magpie useful in filling the gaps in their toolset?
\end{enumerate}
Responses from the survey further allowed us to confirm our target demographic (figure 1.1), find out what type of amenity data  is accessed (figure 1.2) and why current tools are unsatisfactory (figure 1.3). These responses also cemented the need of Magpie (figure 1.4).
\begin{figure}[h]
    \centering
    \fbox{\includegraphics[width=0.5\textwidth]{Figures/fig1.png}}
    \caption{Target user sectors}
    \label{fig:yourlabel}
\end{figure}
\begin{figure}[h]
    \centering
    \fbox{\includegraphics[width=\textwidth]{Figures/fig4.png}}
    \caption{Magpie Usefulness among different users}
    \label{fig:yourlabel}
\end{figure}
for both casual users (User A) \& professionals who access amenity data (User B). The survey also helped us implement additional features \hl{(figure C)} prior to the user evaluation such as a dashboard with search functionality and filters.\\ \\
Magpie has remedied the first issue of fragmented information on amenities, and through this user evaluation, we hope to address the second issue which is making the access to this information easy, quick \& accessible.

\chapter{Experimental Methods}
The goal of the user evaluation is to gain feedback from real users, learn if Magpie works as expected and assess how user-friendly it is. We will be using 2 main methods to collect both qualitative data through open-ended questions, and quantitative data from multiple choice questions from which we will derive insights to improve the Magpie user experience.
\section{Usability Testing}
\subsection{Uncontrolled (Remote)}
\begin{enumerate}
    \item \underline{Objective:} Evaluate user experience in an uncontrolled environment, assess overall functioning of Magpie \& identify any potentional usability issues
    \item \underline{Conditions:} Online followed up by survey
    \item \underline{Methodology:} Reach out to users in our target sectors who know nothing about Magpie, ask them to participate in user experience survey, \hl{bla bla bla}.
    \item \underline{Baseline \& Evaluation metrics:} No baseline; user experience will be evaluated through a satisfaction survey \hl{figure D}.
\end{enumerate}
\subsection{Controlled (Remote)}
\begin{enumerate}
    \item \underline{Objective:} Analyze how users interact with Magpie in a controlled environment guided by specific instructions and tasks to complete \& identify any potentional usability issues
    \item \underline{Conditions:} Online through remote usability tools (SessionReplay/Contentsquare)
    \item \underline{Methodology:} Reach out to users who left their contact email in the market research survey, propose for them to participate in user evaluation survey, conduct the experiment, \hl{bla bla bla}.
    \item \underline{Baseline \& Evaluation metrics:} No baseline; user experience will be evaluated on rate of completion of tasks, difficulties encountered when completing tasks
\end{enumerate}
\subsection{Think-Aloud Protocol (In-person)}
\begin{enumerate}
    \item \underline{Objective:}
    \item \underline{Conditions:}
    \item \underline{Methodology:}
    \item \underline{Baseline \& Evaluation metrics:}
\end{enumerate}
\subsection{Field test (Remote)}
\begin{enumerate}
    \item \underline{Objective:}
    \item \underline{Conditions:}
    \item \underline{Methodology:}
    \item \underline{Baseline \& Evaluation metrics:}
\end{enumerate}

\section{Expert Review}
% Details of the third method

\chapter{Conclusions}
% Add your conclusions here

% References & Table of Figures Page
\newpage
\addcontentsline{toc}{chapter}{References}
\chapter*{References}
% Use \bibliography{yourbibfile} with a .bib file if available
\begin{itemize}
    \item Reference 1
    \item Reference 2
          % Add other references here
\end{itemize}

\newpage
\addcontentsline{toc}{chapter}{Table of Figures}
\listoffigures

\end{document}
