\documentclass{beamer}

% Theme for the presentation
\usetheme{Madrid}  % You can change the theme (e.g., AnnArbor, Berlin, etc.)

% Optional colors
\usecolortheme{dolphin}  % You can change this as well

% Packages to include if needed
\usepackage{graphicx} % For including images
\usepackage{amsmath, amssymb} % For mathematical symbols
\usepackage{hyperref} % For clickable links

% Title Page Info
\title[Weekly Presentation]{Weekly Presentation}
\subtitle{Group 3}
\author{Jessica Fornetti, Anais Blenet, Saul Burgess, Kaustubh Trivedi, Yuanshuo Du, Andreas Kraus}
\institute{TU Dublin}
\date{23rd of September, 2024} % You can specify a fixed date if needed

% Begin document
\begin{document}

% Title Slide
\begin{frame}
    \titlepage{}
\end{frame}

% Section 1
\section{Introduction}

% Use cases & users Slide
\begin{frame}{Use cases and Users}
    \item We will be focusing on parking amenities.
    \item{Potential users and their associated use cases}
    \begin{itemize}
        \item{\textbf{1. Civil servants}} - for urban planning purposes such as:
        \begin{itemize}
            \item ensuring parking spaces adhere to local zoning regulations
            \item assessing the demand/supply for parking in various areas of a city
            \item overseeing existing parking amenities, conditions and pricing
            \item considering environmental and traffic impact when planning to build new spaces
        \end{itemize} 
        \item{\textbf{2. Private Individuals}} - for personal use in the following scenarios:
        \begin{itemize}
            \item viewing real-time parking availability given their current location or a specified destination within a defined radius
            \item  seeing the cost of available parking spots  nearby allowing users to compare and choose the best option
            \item viewing available parking based on special filters, such as handicap-accessible parking, electric vehicle charging stations, etc...
        \end{itemize} 
    \end{itemize}
\end{frame}

% Section 2
\section{Main Content}
\subsection{Approaches}
% Different Approaches Slide
\begin{frame}{Different Approaches}
    \begin{itemize}
        \item{Machine Learning Techniques : Convolutional Neural Networks}
        \item{Integration of Geospatial Data with Public Records}
        \item{Web Based Applications}
    \end{itemize}
\end{frame}

% Different Approaches Slide 2
\begin{frame}{Different Approaches: Convolutional Neural Networks}
    \begin{itemize}
        \item{Data Preparation : Preprocessing Steps}
        \begin{itemize}
        \item{Image Segmentation}
        \item{Noise Reduction}
        \item{Normalization}
        \end{itemize}
        \item{Model Training }
        \begin{itemize}
        \item{Different architectures}
        \item{Hyperparameter Tunning}
        \item{Regularization Techniques}
        \end{itemize}
        \item{Validation and Testing}
    \end{itemize}
\end{frame}

% Different Approaches Slide 3
\begin{frame}{Integration of Geospatial Data with Public Records}
    \begin{itemize}
        \item{Data Fusion}
        \item{Geospatial Analysis }
        \begin{itemize}
        \item{Measure Distances}
        \item{Identify Service Zones}
        \item{Identify Facility Clusters }
        \end{itemize}
        \item{GIS Tools}
    \end{itemize}
\end{frame}

% Different Approaches Slide 4
\begin{frame}{Web Based Applications}
    \begin{itemize}
        \item{User Interface Design}
        \item{Back-End Development}
        \begin{itemize}
        \item{Create Database}
        \item{Preprocess datasets}
        \item{Implement Machine Learning models}
        \end{itemize}
        \item{Scalability and Security}
    \end{itemize}
\end{frame}


% Design & build: Phase 1 Slide 6
\begin{frame}{Design and Build: Phase 1 - Model training}
    \begin{itemize}
        \item{\textbf{1. Data Collection and Preparation} - } High resolution aerial photographs will be gathered from multiple public sources. These images will undergo preprocessing to enhance their quality and consistency such as noise reduction, normalization, and image segmentation.
        \item{\textbf{2. Model Development and Training} - } A CNN will be trained on the preprocessed image to recognize street-level parking areas. Different architectural configurations will be used and hyperparameter tuning will be performed to optimize the model's accuracy and reduce overftitting.
        \item{\textbf{3.Validation and Testing} - } The model will be tested using a separate validation dataset ensuring it performs well in different urban settings. Various performance metrics (accuracy, precision, recall) will be used to assess the model.
    \end{itemize}
\end{frame}



\end{document}
