\documentclass[10pt]{article}
\usepackage[utf8]{inputenc}
\usepackage{amsmath, amssymb}
\usepackage{embedfile}
\usepackage{graphicx}
\usepackage[colorlinks=true, linkcolor=blue, urlcolor=blue, citecolor=blue]{hyperref}
\usepackage{ragged2e}
\usepackage{bookmark}
\usepackage{booktabs}
\usepackage{enumitem}
\usepackage{biblatex} %Imports biblatex package
\addbibresource{sources.bib} %Import the bibliography file

% Metadata
\title{Assignment Title}
\author{Your Name \\ Student ID}
\date{\today}

% To create a separate title page and a table of contents
\begin{document}

% Title Page
\begin{titlepage}
	\centering{

		\vspace{1cm}
		\includegraphics[width=0.4\textwidth]{TU_Dublin_Logo.svg.png} % Example image

		\vspace*{1cm}
		\LARGE\textbf{Systems Architecture}

		\vspace*{1cm}
		\LARGE\textbf{Assignment 2}

		\vspace*{1cm}
		\large\textbf{TU059}

		\vspace*{1cm}
		\large\textbf{Saul Burgess: C19349793}

		\vspace*{1cm}
		\large{School of Computer Science \\ Technological University, Dublin}

		\vspace*{1cm}
		{\large\today}
	}
\end{titlepage}

{\tableofcontents}
{\newpage}

\section{Presentation}

\embedfile{./docs/presentation.pdf}
\subsection{Embedded Files}
\begin{center}
	\embedfile{./docs/presentation.pdf}
	\href{run:./presentation.pdf}{presentation.pdf}

	\vspace*{1cm}

	\embedfile{./SYSARCH_AS2.mp4}
	\href{run:./SYSARCH_AS2.mp4}{video\_of\_presentation.mp4}
\end{center}

\subsection{Azure Blob Storage Files}
\begin{center}
	\href{https://c19349793test.blob.core.windows.net/assignment-share/presentation.pdf}{presentation.pdf}

	\vspace*{1cm}

	\href{https://c19349793test.blob.core.windows.net/assignment-share/SYSARCH_AS2.mp4}{video\_of\_presentation.mp4}

	\vspace*{1cm}

	\href{https://c19349793test.blob.core.windows.net/assignment-share/SYSARCH2_SaulBurgess.pdf}{SYSARCH2\_SaulBurgess.pdf}

\end{center}
{\newpage}


\section{Identity Management Technologies for TU Dublin}
This review provides an overview of Identity Management (IDM) systems, which are
designed to make ensure that the right individuals have access to the
appropriate technology resources within an organization(In this case, TU
Dublin). IDM systems achieve this through a combination of technologies that
manage user identities, their authentication, authorization, and the roles they
are assigned within and without network.\cite{loginradius2023} The focus here is
on cloud-based IDM solutions, addressing the inherent considerations that TU
Dublin must consider when implementing these systems.
\vspace*{0.5cm}

\subsection{Cloud-Based Identity Management}
With the increasing adoption of cloud computing, identity management solutions
have migrated from on-premises installations to cloud-based models. These
solutions, known as Identity-as-a-Service, offer advantages such as
scalability and accessibility. However, they also introduce their own risks,
particularly related to data security.\cite{secureauth2023}
\vspace*{0.5cm}

\subsection{Challenges and Risks}
Some examples of risks associated with cloud-based IDM include:
\begin{description}[leftmargin=!,labelwidth=0pt,itemindent=0pt]
	\item[] \textbf{Data Security:} Storing sensitive identity data on external
	      servers increases vulnerability to breaches.\cite{ahmadi2024cloud}
	\item[] \textbf{Privacy Concerns:} Managing personal data across borders
	      raises concerns under various privacy laws and regulations, such as
	      GDPR.\cite{gdpr-info}
	\item[] \textbf{Dependency on Service Providers:} Reliance on third-party
	      providers can lead to issues of control and
	      availability.\cite{secureauth2023}
\end{description}
\vspace*{0.5cm}

\subsection{Ethical Considerations}
The ethical implications of deploying IDM in the cloud could include:
\begin{description}[leftmargin=!,labelwidth=0pt,itemindent=0pt]
	\item[] \textbf{Data Sovereignty:} The location where data is stored can
	      conflict with national laws, for example GDPR prevents data transfer
	      outside the EU.\cite{gdpr-info}
	\item[] \textbf{User Consent and Transparency:} Users must be informed about
	      how their data is being used and by whom, this is not always possible
	      in the cloud.\cite{ahmadi2024cloud}
\end{description}
{\newpage}

\subsection{Identity Management at TU Dublin}
For TU Dublin, implementing IDM technologies must align with institutional goals
and regulatory compliance. Especially given TU Dublin is within the EU. Some
strategies might include:
\begin{description}[leftmargin=!,labelwidth=0pt,itemindent=0pt]
	\item[*] Ensuring clear data governance policies that comply with EU data
	      protection standards.
	\item[*] Regularly auditing and reviewing access controls and adherence to
	      ethical guidelines.
	\item[*] Selecting IDaaS providers with robust security certifications.
\end{description}
\vspace*{0.5cm}

\subsection{Conclusion}
As TU Dublin considers expanding its cloud infrastructure, utalising IDM
solutions is vital for guarding digital identities and maintaining trust between
students and IT administration. Addressing the associated risks and ethical
concerns is a necessity for the successful deployment of these technologies.
	{\newpage}


\section{Comparative Analysis of Cloud-Based Identity Management Solutions}
This section offers a comparative analysis of cloud-based identity management
offerings, specifically DUO, Microsoft, and Tenfold. We explore how these
services utilize Cloud Service Oriented Architecture (hereby SOA) and assess
their impacts on privacy and security within education.\\ \\
Cloud-based identity management systems are a necessity for securing access to
educational resources. They allow institutions to manage digital identities ,
ensuring that users have appropriate access rights based on their roles. These
systems are particularly important in educational settings, where a variety of
users require access to different resources.\cite{loginradius2023}

\subsection{Cloud Service Oriented Architecture (SOA)}
SOA in cloud services provides a flexible and reusable framework that
facilitates integration and scalability. Each of the considered systems—DUO,
Microsoft, and Tenfold—employs SOA principles to offer services that can be
customized for specific entity needs, be it business or otherwise.\cite{aws_soa}

\subsection{DUO Security}
DUO Security uses a multi-factor authentication model that integrates with many
services, including VPNs and cloud applications. This makes it a versatile
solution for educational institutions with diverse needs.\cite{duo_docs}

\subsection{Microsoft Azure Active Directory}
Microsoft's solution emphasizes interoperability with Microsoft apps, including
Office 365 and Azure. This makes it an attractive option for institutions
already using Microsoft services. However, it may not be as flexible for those
using other platforms. \textit{Note, it is now called Entra ID, not Active
	Directory}\cite{microsoft_entra}

\subsection{Tenfold Security}
Tenfold offers a comprehensive identity management solution that integrates with
a wide range of services. It supports LDAP and SAML, making it a versatile
option for institutions with diverse IT environments. However, it may be more
complex to set up than other solutions. \textit{DUO does support LDAP, but not
	as well as Tenfold}\cite{tenfold_docs}
{\newpage}

\subsection{Comparative Analysis}
The below table is based on the features of the three systems, DUO, Microsoft,
and Tenfold. Not user experience, or data, but rather the tools that each system
offers.\\

\begin{table}[h]
	\centering
	\begin{tabular}{@{}lccc@{}}
		\toprule
		Feature         & DUO      & Microsoft & Tenfold  \\ \midrule
		Ease of Use     & High     & Moderate  & Moderate \\
		Scalability     & Moderate & High      & High     \\
		Customizability & Low      & High      & Moderate \\
		Security        & High     & Very High & High     \\
		\bottomrule
	\end{tabular}
	\caption{Comparative features of identity management systems}
	\label{tab:comparison}
\end{table}

\subsection{Considerations}
Identity management systems in educational settings raise significant concerns.
Students, and education at large, has a high burden of privacy and security.
Ethically, the management of student and faculty identities must respect user
consent and minimize data collection to necessary levels. Privacy is paramount,
with a need to comply with laws such as GDPR \cite{gdpr-info} in Europe and
FERPA\cite{usdoe_ferpa} in the US. Security, while ensuring data protection, must also prevent
unauthorized access, a critical aspect given the sensitivity of educational
data.

\subsection{Impact on Education}
These systems can streamline access to educational resources, but must be
implemented with caution to avoid complicating access to educational resources.
Educators and IT administrators must work together to ensure these tools are
used efficiently, and not overused.

\subsection{Conclusion}
Cloud-based identity management systems, when implemented within the framework
of SOA, provide robust, scalable, and flexible solutions that can enhance the
security of educational institutions. However, careful considertaion must be to
the implications of implementing these systems.

	{\newpage}

\section{Summary and Reflection: Optimization of ELT Process}
This section provides a summary and reflection on the research article titled
\begin{center}
	\textit{ "Extract-Load-Transform (ELT) Process Runtime Analysis and
		Optimization by Aleksei E. Zvonarev and colleagues." }\cite{zvonarev2023elt}
\end{center}
The study explores methods to enhance the efficiency of the ELT process in data
handling operations.

\subsection{What is ELT?}
Extract-Load-Transform is a data processing methodology that is designed
to fetch data from various sources and load it directly into a data warehouse or
repository without prior transformation. This approach contrasts with the more
traditional Extract-Transform-Load (ETL) process, where data is transformed
prior to being loaded into the data warehouse. The main advantage of ELT is its
ability to leverage the processing power of modern data warehouses, allowing for
the transformation of data to be handled more efficiently.

\subsection{Summary}
The article presents an approach for optimizing the transformation phase of the
ELT process. Recognizing that traditional sequential execution of data
transformation can be time-consuming, the authors propose and compare two
algorithms for optimization: a basic parallelization approach and an enhanced
version that further parallelizes dependent procedure blocks.
\\ \\
Through experimentation, it was determine that the enhanced parallelization
algorithm considerably reduces the execution time compared to the basic
parallelization and traditional sequential approaches.

\subsection{Reflection}
The significance of the study lies in its potential application in various
real-world scenarios where there is large volumes of data. This article provides
valuable insights for entities looking to improve their data warehousing
performance, underscoring the shift from conventional methods to more efficient
and scalable solutions.
\\ \\
Furthermore, the research exemplifies a growing trend towards integrating
various technologies to solve complex problems in the field of data analytics.
The findings may encourage further research into the application of similar
methodologies across different database management systems, potentially leading
to more universally applicable optimization techniques.

	{\newpage}

\section{Reflection on Azure}
This section reflects on my experience in setting up hyperlink accessible files
on Azure.

\vspace*{0.5cm}

\subsection{Process Overview}
I started by creating an Azure Storage Account through the Azure Portal. The
guided setup was intuitive and well-documented, making the initial steps easy. I
then created a Blob Container,  I opted for "Blob" access to allow public
accessibility to the files.

\vspace*{0.5cm}

\subsection{Challenges Encountered}
Despite the straightforward setup, I faced several challenges:
\begin{description}[leftmargin=!,labelwidth=0pt,itemindent=0pt]
	\item[] \textbf{Understanding Access Levels}: Choosing between "Container"
	      and "Blob" access levels was a bit confusing at first. I had to do
	      additional research.
	\item[] \textbf{URL Accessibility}: Initially, some of the hyperlinks to the
	      files were not accessible. After some investigation, I found that
	      the issue was due to incorrect Blob properties. Correcting these properties
	      fixed the hyperlink issues.
\end{description}

\vspace*{0.5cm}

\subsection{Learning Outcomes}
This task helped me learn about the capabilities of Azure Blob Storage,
particularly its potential for integration with other Azure services. I also
gained a better understanding of the importance of access levels in cloud
storage and the need for proper configuration to ensure data security and
accessibility.

	{\newpage}
\printbibliography{}
\end{document}
